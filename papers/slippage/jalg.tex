

\subsection{Algorithm 1:  Combinatorial Blocking}

The first algorithm uses additional unmodified delta debugging passes, but
``blocks'' them from producing the same reduced test.  Given test $t$
that reduces to $r$, we compute all subsets of components of $r$,
$C_r$.  The set of reduced tests is then computed by running delta
debugging starting with each $t-c$ that fails, where $c \in C_r$: $c$
is the blocked components of $r$.  So long as even one component is
blocked, it is impossible to reproduce the same $r$.  The intuition
is that to find a reduced failing test that exhibits a different fault
than $r$, we want to run delta debugging in such a way as to produce a
test as different as possible from $r$; ideally we would like a
reduced test sharing no components with $r$.  However, $r$ will likely contain
components that must appear in any failing test:  for example,
calls to {\tt mount} appear in all useful file system tests,
and interesting XML files seldom lack the {\tt <} character.
Therefore, rather than ``blocking'' all components of $r$, we try
delta debugging with different sets of components blocked.

In practice, iterating through all subsets may be too expensive if $r$
is large.  We therefore make an assumption:  if $t-c_1$ does not fail,
and $c_1 \subset c_2$, then $t-c_2$ also does not fail.  The blocking
algorithm begins its search by blocking all single components of $r$,
then proceeds to all combinations of 2 components, etc., at each stage
only considering combinations that contain no combination that did not
yield a failing test.  With this optimization, the expense of blocking
becomes low enough to also apply the approach to the new
reduced tests found at each stage.  To block all previously discovered
reductions, it is neccessary to compute combinations that include at
least one component from each reduced test produced thus far.

 Algorithm \ref{comb-block} shows
the formal definition of the algorithm, which we refer to as {\tt
  comb-block} (combinatorial blocking).  This algorithm depends on a
function {\tt block-all ($T$,$s$)}, which given a set of tests $T$ and
a combination size $s$,  returns all combinations of components of tests $t
\in T$ of size $s$ such that each combination has at least one element
from each $t \in T$.  We omit the definition of {\tt block-all} in
the interests of space.

\begin{algorithm}
\caption{}
\label{comb-block}
\begin{algorithmic}[1]
\Require{failing test $t$, reduced failing test $r$, search depth
  $d$, max combinations to consider $m$}
\State {reductions = \{$r$\}}
\State {handled = $\emptyset$}
\State {notfailed = $\emptyset$}
\State {count = 0}
\While {$d$ > 0}
\State{new = $\emptyset$}
\For {$s$ = 1 to total components in reductions}
\For {$c \in$ {\tt block-all}(reductions-handled, $s$)}
\State {count = count + 1}
\If {count > $m$}
\Return {reductions}
\EndIf
\State {handled = handled $\cup$ \{$c$\}}
\If {$c \not\in$ notfailed}
\If {fails($t-c$)}
\State{new = new $\cup$ \{\tt{ddmin}($t-c$)\}}
\Else
\State{notfailed = notfailed $\cup$ \{$c$\}}
\EndIf
\EndIf
\EndFor
\EndFor
\State{$d$ = $d - 1$}
\State{reductions = reductions $\cup$ new}
\EndWhile
\State {return reductions}
\end{algorithmic}
\end{algorithm}
