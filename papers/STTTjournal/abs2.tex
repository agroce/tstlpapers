A test harness, in automated test generation, defines the set of valid
tests for a system, and usually also defines a set of correctness
properties.  The difficulty of writing test harnesses is a major
obstacle to the adoption of automated test generation and model
checking.  Languages for writing test harnesses are usually tied to a
particular tool and unfamilliar to programmers.  Such languages often
limit expressiveness.  Writing test harnesses directly in the language
of the Software Under Test (SUT) is a tedious, repetitive, and
error-prone task, offers little or no support for test case
manipulation and debugging, and produces hard-to-read,
hard-to-maintain code.  Using existing harness languages or writing
directly in the language of the SUT also tends to limit users to one
algorithm for test generation, with little ability to explore
alternative methods.  In this paper, we present TSTL, the Template
Scripting Testing Language, a domain-specific language (DSL) for
writing test harnesses.  TSTL compiles harness definitions into a
graph-based interface for testing, making generic test generation and
manipulation tools that apply to any SUT possible.  TSTL includes a
compiler to Python as well as a suite of tools for generating,
manipulating, and analyzing test cases.  These include simple but
easily extensible explicit-state model checkers.  This paper showcases
the features of TSTL by focusing on one large-scale testing effort,
directed by an end-user, to find faults in the most widely used
Geographic Information Systems (GIS) tool, Esri's ArcGIS, using its
ArcPy scripting interface.  We also demonstrate TSTL's power for
prototyping novel test generation algorithms.



