\section{Conclusions}
\label{sec:conclusion}

This paper presents the latest version of the TSTL
\cite{NFM15,ISSTA15,tstl} domain-specific language for testing, which
enables a declarative style of test harness development, where the
focus is on defining the actions in valid tests, not determining
exactly how tests are generated.  Because TSTL, inspired by the SPIN
model checker, produces a software-under-test-independent interface
for testing, TSTL makes it possible for users to easily apply
different test generation methods to the same system without undue
effort.  The same approach makes it possible for researchers to rapidly prototype novel test generation
methods, and evaluate them in a context where differences in test
infrastructure not relevant to the algorithms at hand can be minimized.

TSTL has, in the year since its initial introduction,
already been used to discover previously unknown (to our knowledge)
faults in multiple Python libraries, including the very widely-used
ArcPy site package for GIS scripting. 
As future work, we plan to continue to use TSTL to explore novel
testing algorithms, investigate the relative strengths of systematic,
stochastic, and directed test generation methods, and apply TSTL to
look for faults in widely used libraries.  Finally, we plan to
port TSTL to additional
programming languages beyond Python and Java, and add automatic
support for new properties, including
information-flow based security checks.

More generally, TSTL takes the approach to GIS embodied in ArcPy, or
to biology embodied in QIIME \cite{QIIME}, and applies it to testing:  TSTL
makes it possible for users to create, manipulate, and execute test
cases in the context of an easily learned programming language.  Just
as ArcPy makes the automation of GIS tasks easier, TSTL aims to make
the automation of all testing tasks easier, by supporting the
functionality common to most testing tasks.  The choice of System
Under Test in TSTL is analogous to the choice of using GIS to analyze
epidemiological data or urban traffic routes using ArcPy; ArcPy
provides features that are common across all such applications.  TSTL
provides the same functionality with respect to software testing.

{\bf Acknowledgments:} The authors would like to thank John Regehr,
David R. MacIver, Klaus Havelund, our anonymous reviewers, and
students in CS362, CS562, and CS569, for discussions related to this work.
A portion of this work was funded by NSF grants CCF-1054786 and
CCF-1217824.
