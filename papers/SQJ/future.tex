\section{Work In-Progress}
\label{future}

While not all of the language and tool changes discussed above have
been introduced in the release version of TSTL, all of these features are at
minimum undergoing testing prior to release.  The features discussed
in this section are in the early, exploratory, design phase.

\subsection{More Complete Correctness Checks}
\label{futureoracle}

One key omission in the ArcPy test harness is the lack of correctness
properties.  This version only fails tests when 1) the system crashes
or 2) an unexpected exception is raised.  Determining the proper
results for complex operation sequences on arbitrary data is
difficult.  To support conditional metamorphic properties \cite{MetaTest}, we plan to
add a TSTL feature to check a property conditional on the set of
actions called.  For example, if fields are never added or deleted, the number of
fields for each feature class should remain constant.  Unfortunately,
the set of properties that are independent of the underlying data used
may be relatively small, or the conditions under which a property
should hold so limited it may be hard to produce highly effective
testing based just on conventional correctness properties or limited
metamorphic testing.

\subsection{Differential Testing within One Version}
\label{sec:reftest}

One way to work around the difficulties of specifying properties of
complex systems is to use differential testing \cite{Differential,ICSEDiff}:  if two systems
implement the same functionality, and are given the same inputs, they
should produce the same output.  This approach can also be applied
within a single version of ArcPy:  because ArcGIS supports more than
one storage format (shapefiles, file geodatabases, and personal
geodatabases), the same data can be stored in multiple formats.  Using
TSTL's support for reference pools, the same operations can be
performed on the same data stored in different formats, and the
results can be checked, including success or failure of ArcPy calls
and spatial properties and field values of the manipulated data.

\subsection{Testing Multiprocessing}

ArcPy scripts can use Python's multiprocessing features to exploit
multicore machines and gain efficiency.  However, this can expose
ArcGIS or ArcPy faults, especially problems with the (not well
documented) locking system.  TSTL can use multiprocessing in actions
(since actions can call arbitrary Python code), but does not include
support for running multiple threads of execution at the TSTL level.
One problem with testing multiprocessing use of ArcPy is that
concurrency is much more likely to produce nondeterministic test
behavior than our existing testing.

%\subsection{Optimizations for Testing}

\subsection{Parallel Testing}

While multiprocessing on a single machine can introduce additional
problems with test determinism and locking interactions, running tests
in parallel using either cloud computing or virtualization on a
single multicore machine could improve test throughput.   This is
possibly more important for ArcPy than in many other testing
scenarios, given the unusually high cost of even single API calls.

\subsection{Alternative Test Generation Techniques}

TSTL is meant to provide a platform for experimenting with various
methods for generating tests \cite{NFM15}.  To date, we have only used
pure random testing, since for our purposes simply generating tests as
quickly and simply as possible has been highly effective.  However, in
the future we plan to explore alternative generation methods.  In some
cases, this should be simple.  For example, swarm testing
\cite{ISSTA12} only requires identifying features \cite{groce2013help}
(which for ArcPy will simply be different API functions) that can be
omitted from tests, to concentrate testing on a particular aspect of
the API.  Various search-based \cite{FA11} and machine-learning-based
\cite{ISSRE} approaches to test case generation can also be implemented
  using TSTL.  The inabilty to save ArcPy memory state and the high
  cost of backtracking via replay makes methods such as BFS and DFS
  exploration \cite{NFM15} that resemble model-checking
  \cite{ModelChecking,ModelDriven} unlikely to pay off, however.
