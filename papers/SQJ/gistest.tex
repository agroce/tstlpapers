%%%%%%%%%%%%%%%%%%%%%%% file template.tex %%%%%%%%%%%%%%%%%%%%%%%%%
%
% This is a general template file for the LaTeX package SVJour3
% for Springer journals.          Springer Heidelberg 2010/09/16
%
% Copy it to a new file with a new name and use it as the basis
% for your article. Delete % signs as needed.
%
% This template includes a few options for different layouts and
% content for various journals. Please consult a previous issue of
% your journal as needed.
%
%%%%%%%%%%%%%%%%%%%%%%%%%%%%%%%%%%%%%%%%%%%%%%%%%%%%%%%%%%%%%%%%%%%
%
% First comes an example EPS file -- just ignore it and
% proceed on the \documentclass line
% your LaTeX will extract the file if required
\begin{filecontents*}{example.eps}
%!PS-Adobe-3.0 EPSF-3.0
%%BoundingBox: 19 19 221 221
%%CreationDate: Mon Sep 29 1997
%%Creator: programmed by hand (JK)
%%EndComments
gsave
newpath
  20 20 moveto
  20 220 lineto
  220 220 lineto
  220 20 lineto
closepath
2 setlinewidth
gsave
  .4 setgray fill
grestore
stroke
grestore
\end{filecontents*}
%
\RequirePackage{fix-cm}
%
%\documentclass{svjour3}                     % onecolumn (standard format)
%\documentclass[smallcondensed]{svjour3}     % onecolumn (ditto)
\documentclass[smallextended]{svjour3}       % onecolumn (second format)
%\documentclass[twocolumn]{svjour3}          % twocolumn
%
\smartqed  % flush right qed marks, e.g. at end of proof
%
\usepackage{graphicx}
\usepackage{code}
\usepackage{xcolor}
%
% \usepackage{mathptmx}      % use Times fonts if available on your TeX system
%
% insert here the call for the packages your document requires
%\usepackage{latexsym}
% etc.
%
% please place your own definitions here and don't use \def but
% \newcommand{}{}
%
% Insert the name of "your journal" with
% \journalname{myjournal}
%
\begin{document}

\title{Extensible and Usable Testing for Geographic Information System Automation}
%\subtitle{Do you have a subtitle?\\ If so, write it here}

%\titlerunning{Short form of title}        % if too long for running head

\author{Josie Holmes         \and
  Alex Groce           \and
  James O'Brien
}

%\authorrunning{Short form of author list} % if too long for running head

\institute{Alex Groce \at
              School of Electrical Engineering and Computer Science\\
              Oregon State University\\
              \email{agroce@gmail.com}           %  \\
%             \emph{Present address:} of F. Author  %  if needed
           \and
           Josie Holmes \at
             Department of Geography\\
             Pennsylvania State University\\
           \email{jdh396@psu.edu}
           \and
           James O'Brien \at
          Risk Frontiers\\
           Macquarie University\\
          \email{James.OBrien@mq.edu.au}
}

\date{Received: date / Accepted: date}
% The correct dates will be entered by the editor


\maketitle

\begin{abstract}
The difficulty of writing test harnesses is a major obstacle to the
adoption of automated testing and model checking.  Languages designed
for harness definition are usually tied to a particular tool and
unfamiliar to programmers; moreover, such languages can limit
expressiveness.  Writing a harness directly in the language of the
software under test (SUT) makes it hard to change testing algorithms,
offers no support for the common testing idioms, and tends to
produce repetitive, hard-to-read code.  This makes harness
generation a natural fit for the use of an unusual kind of domain-specific language (DSL). This paper defines a \emph{template
scripting} testing language, TSTL, and shows how it can be used to
produce succinct, readable definitions of state spaces.  The concepts
underlying TSTL are demonstrated in Python but are not tied to it.


\keywords{Automated testing
  \and End-user developers \and Geographic Information System \and
  Testing languages \and Debugging}
% \PACS{PACS code1 \and PACS code2 \and more}
% \subclass{MSC code1 \and MSC code2 \and more}
\end{abstract}

\section{Introduction}

Software test automation encompasses two challenges: (1) automated
execution and determination of results for human-created tests, and
(2) truly automatic generation of tests.  Both are critical for effective,
efficient software testing, but only test generation offers the
potential to discover faults without human determination that a
particular execution scenario has the potential to behave incorrectly.
Automated generation of tests relies on the construction of \emph{test
  harnesses}.  A \emph{test harness} defines the set of valid tests
(and, usually, a set of correctness properties for those tests) for
the Software Under Test (SUT).  This paper presents a language and
tools applying insights from the world of explicit-state model
checking to the problem of producing test harnesses for automated
test generation, whether tests are produced by a exhaustive
state-space exploration as in model checking, or via less
systematic methods.

Building a test harness is a task that even experts in
model checking and automated testing often find painful
\cite{woda08,woda12}.  The difficulty of harness generation is one
reason for the limited adoption of automated testing and model
checking methods by the typical developer who writes unit tests.  This is
unfortunate, as even simple random testing can often uncover subtle
faults.

The ``natural'' way to write a test harness is as code in the language
of the SUT.  This is obviously how most unit
tests are written, as witnessed by the proliferation of tools like
JUnit \cite{JUnit} and its imitators (e.g., PyUnit, HUnit, etc.).  It
is also how many industrial-strength random testing systems are
written \cite{ICSEDiff,AMAI}.  A KLEE ``test harness'' \cite{KLEE} for
symbolic execution is written in C, with a few additional constructs
to indicate which values are symbolic.  This approach is common in
model checking as well: e.g., Java Pathfinder \cite{JPF,JPF2} can
easily be seen as offering a way to define a state space using Java
itself as the modeling language, and CBMC \cite{CBMC,CBMCp} performs a
similar function in C, using SAT/SMT-based bounded model checking
instead of explicit-state execution.  JPF in particular has shown how
writing a harness in the SUT's own language can make it easy to
perform ``apples to apples'' comparisons of various testing/model
checking strategies \cite{JPFRandTest}.


\begin{figure}[b]
{\scriptsize
\begin{code}
op = choice(operations);
val1 = choice(values);
val2 = choice(values);
if (op == op1 \&\& guard1) \{
    call1(val1);
\} else if (op == op2 \&\& guard2) \{
    call2(val1,val2);
\} else if (op == op3 \&\& guard3) \{
    call3(val1,val2);
...
\end{code}
}
\vspace{-0.15in}
\caption {A test harness in the SUT's language.}
\label{fig:badharness}
\end{figure}

Unfortunately, writing test harnesses this way is a highly repetitive
and error-prone programming task, with many conceptual ``code clones''
(e.g. Figure \ref{fig:badharness}). A user faces difficult choices in
constructing such a harness. For example, the example harness always assigns {\tt val2} even
though {\tt call1} only uses {\tt val1}, to avoid having to repeat the
choice code for calls 2 and 3.  The harness is almost certainly
sub-optimal for  random testing, where the lack of any
memory for previously chosen values can make it hard to exercise code
behaviors that rely on providing the same arguments to multiple method
calls (e.g., {\tt insert} and {\tt delete} for container classes).
The construction of a harness becomes even more complex in realistic
cases, where the tested behaviors involve building up complex types as
inputs to method calls, rather than simple integer choices. For
example, consider the problem of testing a complex Python library.  Figure
\ref{fig:MakeFeatureLayer} shows a portion of the Python documentation
for one function in the ArcPy \cite{ArcPy} site
package for Geographic Information Systems (GIS) automation.  Rather than taking a single integer, this
function call requires complex inputs --- a feature class or
layer, an SQL expression, and other complex types that we can assume
are also difficult to construct.  A harness testing a typical
real-world library must manage the creation of values of many such complex types.
Moreover, because building up function inputs is itself complicated
and requires complex method calls, these values cannot simply be produced on each iteration, but must be
stored and selected for use in future calls.
The code quickly becomes hard to read, hard to maintain, and hard to
debug.  In some cases \cite{AMAI} the code for a sophisticated test
harness approaches the SUT in complexity and even size!  The code's
structure also tends to lock in many choices that would ideally be configurable.

\begin{figure}[t]
{\scriptsize
\begin{code}
MakeFeatureLayer\_management(in\_features, out\_layer, {where\_clause}, {workspace}, {field\_info})
   
   Creates a feature layer from an input feature class or layer file. The layer
   that is created by the tool is temporary and will not persist after the session
   ends unless the layer is saved to disk or the map document is saved.
    
INPUTS:
 in\_features (Feature Layer):
   The input feature class or layer from which to make the new layer. Complex
   feature classes, such as annotation and dimensions, are not valid inputs to this tool.
 where\_clause \{SQL Expression\}:
   An SQL expression used to select a subset of features. For more information on
   SQL syntax see the help topic SQL reference for query expressions used in ArcGIS.
...
\end{code}
}
\caption{Documentation for a function in Esri's ArcPy site package.}
\label{fig:MakeFeatureLayer}
\end{figure}

One of the most important of these locked-in choices is the  test generation
method.  Writing a harness by hand usually makes it hard to try out
new strategies.   Writing novel testing strategies in even such
an extensible platform as Java Pathfinder is hardly a task for the
non-expert.
The harness in Figure \ref{fig:badharness} may support random testing and
some form of model checking, if it is written in Java and can use JPF
or a library for adaptation-based testing \cite{ISSRE12}. Such a
harness will likely be completely inflexible as to generation method if written in Python, C,
or another language without that level of tool support.

What the user really wants is to simply provide a concise version of the information in
Figure \ref{fig:MakeFeatureLayer}, some configuration details (e.g., how many
feature classes to keep track of at once), and then try different test
generation methods.  While some automated testing tools for Java \cite{FA11,Pacheco}
can automatically extract
method signatures from source code and produces tests,  
using such a tool locks a user into one test generation method.
Completely automatic extraction also often fails to handle
the subtle details of harness construction, such as defining guards
for some operations, or temporal constraints between API calls that
are not detectable by simple exception behavior.  Understanding
problems with automatic extraction can be hard with large libraries,
since the extraction tends to either produce internal data structures
only or produces a huge, impenetrable mass of code. The user \emph{wants} a
declarative harness, but often \emph{needs} to program critical details of a
harness, and build understanding of the system by performing harness development in
small, incremental steps.

\subsection{Contributions}

In this paper we describe a complete, Domain Specific Language (DSL)-based approach that combines
a simple means to produce a declarative harness with the full power of
a complete programming language.  TSTL (the Template Scripting Testing
Language) compiles a declarative description of system state and
actions into a library in the language of the System Under Test
(SUT).  This library allows the creation of objects providing an API
for testing the SUT, including support for state comparison,
abstraction, backtracking, automatic test case reduction, code coverage, and support
for sophisticated regression testing.

Using an ongoing case study,
we show how to apply TSTL and its tool suite to a large, real-world software
library used in critical applications.  The test effort has been driven and directed not by a
software testing researcher (as is the usual case), but by a domain
expert in the Geographic Information Systems (GIS) SUT.  In the
course of this effort, multiple faults and undocumented restrictions of the library
under test have been discovered, and the TSTL language and tool suite have
been transformed from a research prototype into a complete system for
software testing.

This paper presents the most complete presentation of the TSTL
language and tools, and we hope that it satisfies three critical goals:

\begin{itemize}
\item First, would-be users wanting to take advantage of automated
  test generation should be able to base their own testing
  efforts using TSTL on the example code in this paper (and that available
  in the TSTL github repository \cite{tstl}).
  This paper thus completely describes the concepts behind TSTL, the
  semantics of the language, and the tools available in TSTL.
  Previous papers on TSTL \cite{NFM15,ISSTA15} reported a much less full-featured version of the
  language using a difficult-to-read syntax.

\item Second, researchers should be able to use the information in this paper to
  extend existing TSTL tools or build their own tools to explore novel
  test generation strategies, automated debugging methods, and other
  research prototypes.  TSTL enables easy comparison of
  methods in a framework reducing the burden of implementation
  and avoiding irrelevant differences in performance due to underlying
  infrastructure.  The growing set of SUTs
  included in the TSTL distribution, which includes large and widely
  used Python libraries, can provide benchmarks for
  experimental efforts.  

\item Finally, unlike previous publications on TSTL, this paper
  emphasizes the fact that TSTL, unlike other testing DSLs or tools,
  at heart transforms a definition of valid tests (and properties) for
  a System Under Test into a \emph{programming language interface} for testing
  that system.  Tests in TSTL are not inaccessible entities internal to
  a tool,
  or only represented as unit tests (i.e., programs) that cannot be
  easily manipulated and analyzed, but first-class objects in the
  language of the System Under Test.  To our knowledge, this approach
  to testing has not been previously explored, and it was not
  emphasized (or even clearly presented) in earlier publications on TSTL.
\end{itemize}


The organization of this paper is as follows.  In Section
\ref{sec:dsltest} we present the basic idea of a DSL for testing, and
distinguish TSTL from other testing DSLs.  Section \ref{sec:arcpy} 
provides background on the ArcPy GIS case study used throughout the
paper.  Section \ref{sec:lang} provides a full description, with
examples, of the core TSTL language and semantics.  Section
\ref{sec:tools} describes the tools included with TSTL, and
Section \ref{sec:build} describes how researchers and developers can
build their own TSTL-based testing tools to support additional
testing, debugging, or regression strategies.  Section
\ref{sec:langext} introduces the novel TSTL concept of making testing
a first-class activity in a programming language, similar to how other
libraries make GIS (ArcPy), scientific computing (NumPy \cite{NumPy},
SciPy \cite{SciPy}) or bioinformatics
(QIIME \cite{QIIME}, Biopython \cite{biopython}, scikit-bio \cite{scikitbio}) activities simple to use in either a scripted or
interactive manner.  Faults discovered
using TSTL, in ArcPy and other systems, are described briefly in Section
\ref{sec:bugs}.  We survey the most closely related work in Section
\ref{sec:related}, and summarize our conclusions in Section \ref{sec:conclusion}.

\section{Domain Specific Languages for Testing}
\label{sec:dsltest}

The nature of test harness construction suggests the use of a
\emph{domain-specific language} (DSL) for testing \cite{ISOLA12}.  DSLs
\cite{Fow10} provide abstractions and notations to support a
particular programming domain. The use of DSLs is a formalization of
the long-standing approach of using ``little languages,'' as advocated by Jon Bentley in a
Programming Pearls column \cite{LitLang} and exemplified in such system
designs as UNIX.  DSLs typically come in two forms: \emph{external}
and \emph{internal}.  An external DSL is a stand-alone language, with
its own syntax.  An internal DSL, also known as a domain-specific
embedded language (DSEL), is hosted in a full-featured programming
language, restricting it to the syntax (and semantics) of that
language.  Many attempts to define harnesses can be seen as internal
DSLs \cite{UDITA,ISSRE12,JPF2,CBMCp,KLEE}.  Neither of these choices
is quite right for test harnesses.  Simply adding operations for
nondeterministic choice still leaves most of
the tedious work of harness definition to the user, and makes changing
testing approaches difficult.  With an external DSL, the user
must learn a new language, and the easier it is to learn, the less
likely it is to support the full range of features needed.

A novel approach is taken in recent versions of the SPIN model checker
\cite{SPIN}.  Version 4.0 of SPIN \cite{ModelDriven} exploited the
fact that SPIN works by producing a C program from a PROMELA model
to allow users to include calls to the C language in their PROMELA models.  The
ability to directly call C code makes it much easier to model check
large, complex C programs \cite{AMAI,ModelCode}.  C serves as a
``DSEL'' for SPIN, except that, rather than having a domain-specific
language inside a general-purpose one, here the domain-specific
language hosts a general-purpose language.  A similar embedding is
used in {\tt where} clauses of the LogScope language for testing Mars
Science Laboratory software \cite{scriptstospecs}.  We adopt this
approach for our own language and embed the general-purpose language (for expressiveness) in a
DSL (for concision and ease-of-use).

The most significant difference between TSTL and other DSLs for
testing and verification, including SPIN, is that most such systems
are primarily intended to be used as stand-alone tools.  Whether model
checkers \cite{SPIN,JPF2}, model-based testing tools \cite{Taxonomy},
or random testing tools \cite{Pacheco}, these systems are primarily
designed as ``things to \emph{run on} the system under test.''  TSTL
can operate in this manner, but at heart it transforms a definition of
valid tests into a library for creating, executing, manipulating, and
analyzing test cases.  An experienced TSTL user can interact with TSTL
at an interactive command prompt in the language of the SUT, creating,
saving, and modifying tests on-the-fly.  TSTL tools are simply
scripted formalizations of this mode of use, automating repetitive
tasks.  Such an approach is not possible with any other tool of which we
are aware.  Many tasks that are constrained to the functionality provided
by  tools included in other systems (e.g., replay of regression tests) in
TSTL are simplified and made flexible by this approach.

\subsection{TSTL: The Template Scripting Testing Language}

TSTL is based on understanding a test harness as a declaration
  of the possible actions the SUT can take, where these actions are
defined in the language of the SUT itself, with the full power
of the programming language to define guards, perform
pre-processing, and implement oracles.  Our particular approach is
based on what we call \emph{template scripting}.

The \emph{template} part of the name captures the fact that our method
proceeds by processing a harness definition file to output code that
enables testing, much  as SPIN processes PROMELA/C.  The harness
description file consists of fragments of code in the SUT's language
that are expanded, via the TSTL compiler, into a class that allows
an independently written test generation or manipulation tool to
generate, execute, or replay tests, without knowing
any details of the SUT.
A TSTL harness defines a \emph{template} for action definition, and
the compiler 
instantiates the template exhaustively.  The \emph{scripting} aspect indicates
TSTL is designed to be very lightweight and as easy for
users to pick up as a popular scripting language.  TSTL
works best when the SUT language is very
concise, like most scripting languages, making ``one-liners'' of action
definition possible; our initial implementation \cite{tstl} is therefore for
Python\footnote{We also have released a beta version of TSTL for Java \cite{TSTLJava}, to show
 that testing code in non-scripting languages is also possible.}.

\begin{figure}
{\scriptsize
\begin{code}
@from arcpy import *
\vspace{0.1in}
pools:
  <fc> 3 CONST             \# A feature class contains only lines, points, or polygons
  <newlayer> 3 CONST
  <op> 2 CONST 
  <val> 2 CONST
  <whereclause> 2 CONST    \# SQL clause to limit objects in new layers
  <fieldname> 2 CONST      \# Extracted from the shape files
  <fieldlist> 2
\vspace{0.1in}
actions:
\vspace{0.1in}
<fc> := <["d1.shp", "d2.shp", "d3.shp"]>  \# Just shapefiles for this example
<newlayer> := <["newl1", "newl2", "newl3"]>
\vspace{0.05in}
\{IOError\} <fieldlist> := ListFields(<fc>) \# Extract fields from a feature class
len(<fieldlist,1>) >= 1 -> <fieldname> := <fieldlist> [0].name 
<fieldlist> = <fieldlist> [1:] \# Skip to next field
\vspace{0.05in}
<op> := <[">", "<", "<=", ">=", "=", "!="]>
<val> := <1..10>
<val> = <val> * 10
<val> = <val> + 1

<whereclause> := '"' + <fieldname> + '" ' + <op> + str(<val>)
<whereclause> = <whereclause> + ' AND ' + <whereclause>
<whereclause> = <whereclause> + ' OR ' +  <whereclause>
<whereclause> = 'NOT' + <whereclause>

\{ExecuteError\} MakeFeatureLayer\_management(<fc>,<newlayer>)

\{ExecuteError\} MakeFeatureLayer\_management(<fc>,<newlayer>,where\_clause=<whereclause>)
\end{code}
}
\caption{A small TSTL file to test one ArcPy function.}
\label{fig:makefeaturelayer}
\end{figure}

\begin{figure}
{\scriptsize
\begin{code}
import sut, random, time
rgen = random.Random()
sut = sut.sut()
NUM\_TESTS = 1000
TEST\_LENGTH = 200 
for t in xrange(0,NUM\_TESTS):
   sut.restart()
   for s in xrange(0,TEST\_LENGTH): 
       action = sut.randomEnabled(rgen)
       r = sut.safely(action)
       if len(sut.newBranches()) > 0:
          print time.time(),'NEW BRANCHES:', sut.newBranches()
       if (not r) or (not sut.check()):
          pred = sut.failsCheck if r else sut.fails
          print 'TEST FAILED:', sut.error() 
          R = sut.reduce(sut.test(), pred)
          N = sut.normalize(R, pred) 
          sut.generalize(N,pred)
\end{code}
}
\caption{A simple random tester using the interface provided by TSTL.}
\label{fig:rt}
\end{figure}

Figure \ref{fig:makefeaturelayer} shows a simple TSTL harness for the
function documented in Figure \ref{fig:MakeFeatureLayer}.  Even this
short harness supports constructing SQL where clauses of arbitrary
length and selecting field names based on data files.  Figure
\ref{fig:rt} shows a simple pure random test generator that can test
any SUT (including this one) with a TSTL-defined harness.  This
harness, in 20 lines of code, not only provides automated test
generation, but continuous reporting of incremental branch coverage,
delta-debugging \cite{DD} for reduction of failing tests, and
additional TSTL-specific post-processing that further reduces the size
and complexity of test cases for debugging.  The brevity of the test
generator, no matter how complex the SUT, is made possible by the
common functionality of all TSTL-generated testing interfaces.  The
TSTL compiler produces a Python (or other target language) class that allows a test generation or manipulation
tool to view a testing problem as exploration of a (possibly infinite)
graph of states.  Transitions in the graph are the available test actions, executed in the underlying language, and are guarded by both TSTL
restrictions on the semantics of valid tests and user-defined guards
on system behavior.  States include both the (possibly unknown) state
of the SUT and the TSTL state, including pools of values to be used in actions.

In this simple example, the only ``oracle'' is the implicit property
that the system should neither crash nor raise an unexpected
exception.  For testing many systems, this is sufficient: we have
discovered real bugs in many Python libraries with only this level of
checking, similar to most of what a tool such as Randoop
\cite{Pacheco} or JCrasher \cite{CsallnerS04} checks.  TSTL also
checks arbitrary assertions/invariants defined in the language of the
SUT, supporting traditional property-based testing
\cite{ClaessenH00,hypothesis} (described in Section \ref{sec:property}).  Finally, TSTL includes sophisticated
support for differential testing \cite{Differential,ICSEDiff}, where a
system is tested with respect to the behavior of a reference library.
TSTL makes it easy to wrap a reference system to account for expected
differences, and supports partial reference testing (see Section \ref{sec:differential}).

\section{Motivating Case Study: Esri ArcPy}
\label{sec:arcpy}

Esri is the single largest GIS software vendor, with about 40\% of
global market share.  Esri's ArcGIS tools are extremely widely used
for GIS analysis, in government, scientific research, commercial
enterprises, and education.  Automation of complex GIS analysis and
data management is essential, and Esri has long provided tools for
programming their GIS software tools.  The newest such method,
introduced in ArcGIS 10.0, is a Python site-package, ArcPy
\cite{ArcPy}.  ArcPy is a complex library, with dozens of classes and
hundreds of functions distributed over a variety of of toolboxes.
Most of the code executed in carrying out ArcPy functions is the code
for the ArcGIS engine itself.  This source code, written in C++
(amounting to millions of lines), is
not available.  The source code for the latest version (10.3) of the
Python site-package alone, however, which interfaces with the ArcGIS
engine, is over 50,000 lines of code.  This is a very large system
(especially given the compactness of Python code), comparable in size
to the largest software systems previously tested using automated test
generation, such as core Java and Apache libraries
\cite{FA11,Pacheco}.

In order to improve the reliability of ArcPy, we are developing a
framework for automated testing of ArcPy itself, as well as libraries
based on ArcPy.  The TSTL harness for ArcPy is already more than six times as
large as the next-largest such definition previously implemented in
TSTL, even though the harness so far only includes a small portion of ArcPy
API (Application Program Interface) calls. The first stage of testing has resulted in discovery of
multiple faults in ArcPy/ArcGIS, and has required modifications to the
TSTL language and, especially, to the tool chain supporting test replay,
debugging, and test case understanding.

%One of the contributions of this paper is a more in-depth discussion
%of the problems, challenges, and tool utility aspects of testing
%software than is typical in most research papers in the field.  Such
%papers are (understandably) typically focused on novel algorithms or
%empirical evaluations of known methods, rather than the practical aspects of finding
%and understanding faults in a real-world software system.


%\subsection{Automated Testing for the Rest of the World}

Previous work on automated test generation for APIs has been largely carried
out by software testing researchers only, or (at most) by software
testing researchers working with individuals who are primarily
software developers.  This paper describes TSTL in the context of a
testing effort largely directed (and coded) by the first author, who is not a
software developer by profession or education, but a GIS analyst.
The problem of end-user testing
\cite{burnettEUSE,Silos,rothermelTOSEM} is long-standing.  Previous
work in the field has often focused on
non-traditional programming: e.g. spreadsheets
\cite{rothermelTOSEM}, visual languages, or machine-learning systems
\cite{OnlyOracle}.  TSTL is partly designed to allow a user who is
familiar with a software library but not expert in software testing
techniques to test a traditional software API library.  In one sense,
this is a less difficult scenario than spreadsheets
or visual forms, in that the testing is directed by an individual used
to writing and thinking about code.  The concepts in
automated software testing are most easily understood by those
who are also familiar with a conventional
programming language.  On the other hand, ArcPy is
not a small user-developed program but a large, complex system.
ArcPy was also not written by the end-user, or by any of the authors
of this paper, nor have the authors received any assistance in the
effort from Esri.

Automated testing systems more advanced than a simple hand-written
loop generating a few random inputs to a handful of functions, or more
complicated to use than a fully push-button system are often
considered too difficult for practical use even by software developers
or software QA staff \cite{ISSRE12}. Even ``push-button'' tools for
automated testing are sometimes difficult for expert users to install,
apply, and configure \cite{AMAI,CFV08,ISSRE12}.  TSTL aims to be
relatively easy to use for anyone familiar with basic Python
development.  By avoiding the use of a toy problem and presenting TSTL
in the context of a more typical real-world system (vs. e.g., a simple
container class), we hope to make it easier to apply to other
real-world systems.
%One goal of this work
%has been to mature the TSTL language and tool chain so Python
%programmers from all backgrounds can easily apply it to their
%automated testing problems, with tools approximately as easy to use as
%other Python developer tools.

%The second major contribution of this paper is therefore a presentation of an
%approach to automated testing that has been chosen by a GIS analyst, not a
%software developer or testing researcher.  Moreover, we present this
%paper as a proof-of-concept that modern automated testing, even in a
%highly interactive, non-push-button form, can be used by a motivated
%domain expert, with the support of  a domain-specific language
%\cite{Fow10} and a set of tools for generating, analyzing, and replaying tests.



\section{Related Work}

Chen et al. introduced the idea of slippage in the course of
describing efforts to automatically detect different faults in a large
set of failing test cases \cite{PLDI13}.  Hughes et
al. \cite{FindMoreBugs} proposed a modification of QuickCheck
to avoid re-producing known bugs that (in theory)
could mitigate the problem of slippage, but is not directly comparable
to our approach.  The approach of Hughes et al. requires
interpretation of test components (e.g. method calls), and analysis of
patterns, while our approaches are purely algorithmic, with no
additional requirements beyond those of delta debugging itself
\cite{DD}.  It is not clear how best to apply such an approach
 to cases such as {\tt jsfunfuzz} where each component is not a
method call but essentially an arbitrary string, without significant
user effort to define abstractions of components.

There are also approaches that sidestep slippage by initially
producing short test sequences (e.g. recent work by Mao et
al. \cite{Mao}).  However, for many generation algorithms
longer sequences are essential for good fault detection \cite{ASE08,LongBetter}.
\section{A Brief Primer on TSTL}

\begin{figure}
{\scriptsize
\begin{code}
$_{01}$ @import avl
$_{02}$ @import math
\vspace{0.1in}
$_{03}$ <@
$_{04}$ def it(s):
$_{05}$     l = []
$_{06}$     for i in s:
$_{07}$        l.append(i)
$_{08}$     return sorted(l)
$_{09}$ @>
\vspace{0.1in}
$_{10}$ source: avl.py
\vspace{0.1in}
$_{11}$ pool: <int> 4 CONST
$_{12}$ pool: <avl> 2 REF
$_{13}$ pool: <list> 2
\vspace{0.1in}
$_{14}$ log: 1 <avl>.inorder()
\vspace{0.1in}
$_{15}$ property: <avl>.check\_balanced()
\vspace{0.1in}
$_{16}$ <list>:=[]
$_{17}$ ~<list>.append(<int>) 
$_{18}$ <int>:= <[1..20]>
\vspace{0.1in}
$_{19}$ <avl>:=avl.AVLTree()
$_{20}$ <avl>:=avl.AVLTree(<list>)
\vspace{0.1in}

$_{21}$ ~<avl>.insert(<int>) => (len(<avl,1>.inorder()) == pre<(len(<avl,1>.inorder()))>+1) 
  or pre<(<avl,1>.find(<int,1>))>
$_{22}$ ~<avl>.delete(<int>) => (len(<avl,1>.inorder()) == pre<(len(<avl,1>.inorder()))>-1) 
  or not pre<((<avl,1>.find(<int,1>)))>
$_{23}$ ~<avl>.find(<int>)
$_{24}$ <avl>.inorder()
$_{25}$ len(<avl,1>.inorder()) > 5 -> <avl>.display()
\vspace{0.1in}
$_{26}$ reference: avl.AVLTree ==> set
$_{27}$ reference: insert ==> add
$_{28}$ reference: delete ==> discard
$_{29}$ reference: find ==> \_\_contains\_\_
$_{30}$ reference: METHOD(inorder) ==> CALL(it)
$_{31}$ reference: METHOD(display) ==> CALL(print)
\vspace{0.1in}
$_{32}$ compare: find
$_{33}$ compare: inorder
\end{code}
}
\caption{Part of a TSTL definition of AVL tree tests.}
\label{fig:example}
\end{figure}


\begin{figure}
{\scriptsize 
\begin{code}
avl1 = avl.AVLTree()  
int3 = 10  
int1 = 11  
avl1.insert(int1) 
int1 = 1  
avl1.insert(int3) 
avl1.insert(int1) 
int3 = 9  
avl1.insert(int3) 
int2 = 11  
avl1.delete(int2) 
\end{code}
}
\caption{An example TSTL-produced test.}
\label{fig:avlrun}
\end{figure}


TSTL, the Template Scripting Testing Language, \cite{NFM15,ISSTA15,tstl} is a language for defining the
structure of test cases (usually API-call sequences, but also
grammar based tests using string construction), and a set of tools for
use in generating, manipulating, and understanding those test cases.
Figure \ref{fig:example} shows a TSTL definition of tests (known as a
harness definition, or \emph{harness} for short) for a Python class
implementing AVL trees\footnote{AVL trees, named after Georgy
  Adelson-Velsky and Evgenii Landis \cite{AVL} are balanced tree
  structures similar to red-black trees; they are also frequently used as
  examples in software testing \cite{FASE,ISSRE}.} \cite{avltree}, in the latest syntax for TSTL (modified in the
course of the work described in this paper).  Given a harness like the
one in Figure \ref{fig:example}, TSTL compiles it into a Python file
defining a class that gives an interface for testing.  The class
interface provides features such as
querying the set of available testing actions, restarting a test,
replaying a test, collecting data about which lines and branches in
the source code of the SUT have been executed by each test, and other commonly
needed testing features.  The TSTL release \cite{tstl} provides
testing tools that use this interface for test generation and
debugging.

A TSTL test harness defines a set of \emph{pools} that hold values
produced and used during testing.   Using pools \cite{AndrewsTR} is a common
approach to defining API-testing sequences.  The harness also defines a set of
actions that are possible during testing, typically API calls and
assignments to pool values.  Code marked off with the {\tt @}
symbol (lines 1-9) is raw Python code.  The ability to call
arbitrary code in Python makes TSTL extremely expressive.  Other
elements of the ``preamble'' (before action definitions) are the
indication of the source files to be tested (line 10), the information
on how to produce logs of test activity for debugging (line 14), and a
correctness property to check after every test action (line 15).

In this example, there are three pools,
{\tt int}, {\tt list}, and {\tt avl}.  There are four instances of the
{\tt int} pool, which means that a test in progress can store up to 4
{\tt int}s at one time (in variables named {\tt int0}, {\tt int1},
{\tt int2}, and {\tt int3}), and two instances of the {\tt avl} and
{\tt list}
pools.  The actions defined here include setting the value of an {\tt int}
pool to any integer in the range 1-20 inclusive (line 18), setting the value of
an {\tt avl} pool to a newly constructed AVL tree (line 19), and calling an AVL
tree's {\tt insert}, {\tt delete}, {\tt find} and {\tt inorder}
methods (lines 21-24).  One line of TSTL typically defines more than one action. For
example, the line of TSTL code {\tt <avl>.find(<int>)} defines 8 actions, one
for each choice of {\tt avl} and {\tt int} pool instance:  the action set
includes {\tt avl0.find(int0)}, {\tt avl1.find(int0)}, {\tt
  avl0.find(int1)}, and so forth.  


Figure \ref{fig:avlrun} shows a valid test case produced by
running a random test generator on the TSTL-compiled interface
produced by this definition.  TSTL automatically enforces the requirement that
tests are well-formed: for example, no pool instance (such as {\tt
  avl1}) can appear in an action until it has been assigned a value,
and no pool instance that has been assigned a value can be assigned a
different value until it has been used in an action, to avoid
degenerate sequences such as {\tt int3 = 10} followed by {\tt int3 =
  4}.  Each action in a test case is called a ``step'' --- the first
step of the first test case in Figure \ref{fig:avlrun} is storing a
new AVL tree in {\tt avl1}.  Prefacing a ``use'' of a pool with a
tilde {\tt ~} indicates that use does not allow the pool to be
re-initialized; in this example, an AVLtree object cannot be replaced
with a new tree until it has been displayed or traversed.

Figure \ref{fig:example} shows some additional features
of TSTL.  The syntax {\tt guard -> action} allows {\tt action} to take
place only if {\tt guard} is an expression that evaluates to true (see
line 25).  In the example, only AVL trees of size 5 or larger are
displayed.  One feature that is not shown is the syntax for allowing
an action to throw an exception without stopping testing. This
construct, {\tt \{exception [, exception2, ...]\} action}, is used
many times in the TSTL harness, because operations in ArcPy tend to
signal failure by throwing an exception.

Because thus far our testing
has been limited to checking for crash failures and comparing outputs
between versions, we have not needed the TSTL feature for checking an
assertion after an action ({\tt action => pred}, shown on lines 21 and
22), but we expect to use it in future testing, as discussed in Section
\ref{futureoracle}.  The construct {\tt <avl,1>} is a pool use that
always expands to the same pool variable as the indexed occurrence of
the pool in the current action.  The final feature of interest here is the full
support for reference testing.  Marking a pool as {\tt REF} (line 12) creates a
second copy of each pool variable, and transforms the operations
performed on the copy according to a reference mapping, as given in
lines 26-31 of the example code.  In the example, AVL trees are
checked for behavioral equivalence to Python sets.  Lines 32 and 33
also indicate the return values from {\tt find} and {\tt inorder}
should be compared to their reference equivalents.  Section
\ref{sec:reftest} discusses the possibility of using
differential/reference testing \cite{Differential,ICSEDiff} in ArcPy
within a single version.
\section{Language Changes}

The original syntax for TSTL \cite{NFM15} used the {\tt \%} sign to
indicate TSTL constructs and pool variables.  For example, the
assignment of range values to the integer pool in the AVL example
would read:  {\tt \%INT\% := \%[1..20]\%}.  This syntax
produced code that was difficult to read and (in our opinion) unattractive.  The first
author suggested using a notation more similar to that used to
describe the grammar of TSTL itself, enclosing pools in angle
brackets.  On examination, this syntax better reflects the nature of
TSTL pools, since it resembles a BNF (Backus-Naur Form) grammar, and
pools in actions are
conceptually more similar to a grammar non-terminals than simple variables.  Because
C++ and a few other languages use angle brackets for other purposes,
and in order to avoid breaking old harnesses, TSTL continues to allow
the {\tt \%} notation, but future TSTL harnessess for Python will use
the more readable syntax.

Another language improvement inspited by development of the ArcPy harness was
extending the range construct {\tt <[start..end]>} to also express a
list of options, which can be constants or Python expressions.  An
action containing {\tt <[item1, item2, ...]>} expands to multiple
actions, each of which has the literal text of one item.  The
construct {\tt <,[item1,, item2,, ...],>} works the same way, except
that items are delimited by double-commas, allowing the items to even
be partial Python expressions containing commas, for use in,
e.g. constructing function arguments.  The second form is not used in
the ArcPy harness at this time, because it is powerful but somewhat
difficult to read.  In some cases, using multiple lines to define
actions that, in theory, could be handled with a single line is best
for readability reasons (see Section \ref{harness}).

We are also considering, as a result of the experience of developing
the ArcPy harness, moving to a more structured form for TSTL harness definitions.
The current language allows pool definitions, actions, logging code,
raw Python code, and all other TSTL elements to be freely mixed,
without any requirements as to order.  Each line must indicate if it
is not an action definition, with some prefix such as {\tt pool:},
{\tt logging:}, {\tt reference:}, etc.; in practice, however, TSTL
harnessess are always written in an ordered style, with raw code
first, then pool definitions, properties, and logging information,
followed by a long section of action definitions.  Enforcing this
would allow all pool declarations to be prefaced by a single {\tt
  pool:} line at the beginning of the pool definitions, raw Python
code to be contained in a section marked{\tt raw:}, and all other
non-action declarations to be handled in the same way. 

There is also a need for richer structure to avoid repeated
elements in action definitions.  For example, in the TSTL harness, 36
actions allow the {\tt arcpy.ExecuteError} exception to be raised,
which has to be stated for every individual action, and to avoid some
faults a large number of actions may eventually be disallowed for
feature classes or layers with active cursors\footnote{The reader
  not familiar with GIS terminology is directed to the ArcPy
  documentation \cite{ArcPy} and Esri's GIS dictionary \cite{GISDict}.}.  Introducing nested
action groups, which can share guards, allowed exceptions, and
post-conditions could make reading complex TSTL code easier.  We are
currently working to define these language changes in a way that does
not break existing TSTL code.  Discovering the need for this kind of
feature without testing a system as complex as ArcPy would be
difficult.

\section{Tool Changes}

Rather than major algorithmic innovations, the major changes to the
TSTL system required to test ArcPy were fundamentally engineering
challenges.  Only the last two changes described in this section are
concerned with novel testing algorithms and support a research
agenda.  One element that carried across all of these concerns was
fixing bugs in TSTL.  TSTL has been used in several university classes
at the graduate and undergraduate level, and used fairly extensively
in testing research, but a large number of
significant faults went undetected until we attempted to use TSTL to
test ArcPy.  One change made as a result of these problems is that we
now use TSTL to test TSTL's own API.  TSTL faults included subtle
errors in the system for generating unique names for actions, problems
with test replay in some unusual situations, and some faults related
to collecting and analyzing code coverage\footnote{The authors are
  greatful to Ned Batchelder for fixing some related faults in the
  coverage.py \cite{Coveragepy}  Python library for code coverage, at our request.}.

\subsection{Sandboxed Test Case Execution}

Previous use of TSTL had been limited to testing systems where failure
resulted in an uncaught exception or a bad return value from a call,
at worst.  With ArcPy, however, it is very common for a failure to
cause a crash, killing not only ArcPy but the Python environment
running the test case.

We added two features to TSTL's test generators to handle this
problem.  First, we modified the random tester to record each action
to a test case log \emph{before the action is performed}, in order to recover a crashing
test after the test generator terminates abnormally.  After further experimentation,
we discovered that recording just the current test case was not
sufficient; some ArcPy failures required maintaining a history of all
executed actions, since the corruption carried across reimports of the
Python module.  This required us to add a new type of special action
to the TSTL interface, the {\tt restart} action.  When a {\tt restart}
appears in a test, it ``restarts'' the test.  The assumption until we
began testing ArcPy was that all actions before such restarts could be
removed from a test case.  

The second new feature required for effective sandboxing was a
function to enable running a TSTL case in its own Python subprocess,
to allow test reduction, normalization, and generalization (see below)
even for crashing test cases.  It was not neccessary to modify the
TSTL interface's API, as the relevant functions already took an
arbitrary function as reduction predicate, allowing us to simply
produce a ``sandboxed'' version of replay and pass that to TSTL's {\tt
  reduce}, {\tt normalize} etc. calls.  The sandboxing in this case is
minimal: the only resource limitation is that the sandboxed execution
has its own Python process and writes to a temporary copy of the
ArcGIS workspace, but in principle the approach can be extended to
allow more restrictive test case jails in TSTL, including execution in
a virtual machine.


\subsection{Standalone Test Case Generation}

TSTL test cases are saved as text files, where each line of the text
file contains a string representation of an action (with a unique
such string ``name'' for each action).  This format is somewhat human
readable, but is not machine-executable without the assistance of the
TSTL interface.  These test cases are also somewhat misleading for
readers, since guards, post-conditions, reference pool actions, and
allowed exceptions do not appear in this format.

Publishing test cases (or submitting them to Esri) is impractical,
however, if it requires use of the entire TSTL toolchain.  Moreover,
making tests only work within TSTL means it is not possible to
experiment with changing test cases in ways that are not in the TSTL
action set.  It is not  even possible to add print statements to help understand
behavior, without using TSTL's complex logging mechanisms.  

In order to address these problems, we implemented a new TSTL utility
that takes a test case stored in TSTL's internal format and produces a
standalone Python file that does not require any TSTL support.  The
standalone test case generator has options to control whether the
generated test case includes actions on reference pools, property
checks, and handling for allowed exceptions (omitted only if the exception does not
actually take place when the test is replayed).  The reason for disabling
the last functionality is interesting:  most test cases that are
stored are minimized \cite{DD} test cases, from which all actions not
necessary to produce a failure or obtain desired code coverage have
been removed.  In most cases, this means that most ArcPy calls are
successful.  Adding code to handle potential exceptions makes
standalone test cases much longer and harder to read.

In the process of producing the standalone test case utility, we
introduced a more readable format for TSTL action names that is closer
to the code in standalone test cases.  This has, to our surprise, made
reading TSTL test cases in tool output, not just in standalone test
cases, much easier.  The new format has been integrated into new TSTL features.

\subsection{Regression Generation}

One difficulty for ArcPy users is ensuring that their existing scripts
and tools work on new versions of ArcGIS.  Each recent major release
(10.2 and 10.3) after ArcPy's introduction has potentially included some changes
in the behavior of API calls.  Detecting when such changes cause a
script to break is difficult.  A first step would be an automatic way
to find when the return values for calls differ between ArcPy
versions.
Because installing multiple versions of ArcGIS on the same system is
difficult or impossible, our method for finding differences relies on
choosing a reference version (10.3 in our current efforts), and
generating a set of standalone tests that 1) cover a large amount of
ArcPy functionality, including invalid inputs to functions and 2)
record the return values and exceptions raised by calls.  These tests
can be run on any ArcPy version, and will report differences between
the tests and version 10.3.  Performing this kind of differential
testing \cite{Differential} on old or new major releases, or across 64
bit and 32 bit versions, is easy.  In the long run, we also want to
enabled TSTL to produce Python 3.0+ code, for use with ArcGIS Pro,
which uses Python 3.4 instead of 2.7.  This has motivated a branch to
TSTL to support Python 2.7 (unfortunately, Python 3.0 is not fully
backwards compatible with earlier versions, and Python 2.7 is still
the most widely used Python).

We generate regression tests using an approach called \emph{quick
  testing} \cite{icst2014,stvrcausereduce}, which takes a set of tests
produced by random testing, and applies a test case reduction
algorithm \cite{DD} to produce smaller tests that have the same code
coverage as the very large, highly redundant, original set of test
cases.  Automatic quick-testing was added to TSTL's random test
generator to support ArcPy testing.  Combined with standalone test
generation, this allows us to produce test cases that can be run on
any version of ArcPy, and explore a large variety of behavior of the
code.  With ArcPy, coverage alone, unlike previous quick testing
efforts, is insufficient to ensure a useful regression test.  Because
coverage only considers the Python behavior of ArcPy (since we do not
have access to the source for the ArcGIS engine), it may group
behaviors that are not similar together.  We added the ability to
combine coverage preservation with preservation of all ArcPy messages
indicating a successful GIS engine operation, after abstracting away
such details as the runtime of the operation, and so forth.

However, just producing these coverage-and-engine-behavior preserving
standalone tests is not sufficient for good version comparison, since
standalone test cases as produced only check for properties defined in TSTL.  An
additional option was added to the standalone test generator, allowing
it to record the actual return values of all calls, the set of
exceptions thrown, the success/failure messages from the ArcPy engine, and so forth to more precisely record a test's
behavior on an ArcPy version.


\subsection{Test Case Normalization and Generalization}

Understanding ArcPy failures without delta-debugging \cite{DD} to reduce the
test cases to a readable size is, essentially, impossible \cite{MinUnit}.  The
typical test case, before reduction, is 600-2,000 steps long.  Even
after delta-debugging, however, once test cases are more
comprehensible in size, and contain no purely extraneous steps,
understanding ArcPy failures is difficult.

To address this problem, as well as other issues (some of which, such
as triaging large numbers of failing tests, are not at present
problems for ArcPy testing), we developed an algorithm to
\emph{normalize} test cases \cite{ICSTnorm}.  This algorithm applies a series of term
rewriting rules to reduce the number of variables in a test case,
reduce the complexity of API calls made, and other modifications.  In
the case of ArcPy, this often also further reduces test case length
beyond what standard delta-debugging can achieve.  For example, of the
first five crashes detected (some of which turned out to be variations
of one underlying problem), normalization reduced the length of the
delta-debugged test case from 19 to 11 steps, from 18 to 14 steps,
from 27 to 20 steps, from 20 to 16 steps, and from 10 to 9 steps.  In
the last case, the one step removed gave important information about
the problem.

In addition to normalization, we found it essential to apply
generalization \cite{SmartCheck,ICSTnorm} to test cases.  This
algorithm, also produced to aid ArcPy testing, is in a sense the
opposite of normalization.  Normalization takes many tests that differ
in unimportant ways and converts them to one, simple, sometimes
canonical (one test per fault) form.  Generalization takes a single
test, and produces annotations that describe how the test could be
modified while retaining the property of interest --- e.g.,
generalization answers such questions as:

\begin{itemize}
\item Could this
constant value be different, and the test still fail?
\item Could these two API calls be swapped in their position in the
  test case, and the test still fail?
\item Could this freshly created object replace this complex,
  much-modified object in this API call, and the test still fail?
\end{itemize}

Together, normalization and generalization have greatly aided our
understanding of complex ArcPy test cases:  normalization provides a
standard structure for failures, and makes constant values as small as
possible. Generalization tells us when these values can be changed,
without altering the disposition of the test.  The faults described in
this paper are all presented as normalized and generalized test cases.

\subsection{Deep State Startup Testing}

One of the most challenging problems in testing ArcPy is that
individual test operations may take much more time than in traditional
API-based testing.  In traditional API testing, it is assumed that
each step of execution takes at most a few seconds.  Performing a
complex GIS analysis such as a Buffer or Intersect, however, can require
many seconds, or even many minutes (the largest time for a single
operation during a test we have seen is almost 30 minutes).  If some faults depend
on the interactions of multiple complex analyses, detecting them will
be extremely expensive.  Simply reaching system states that are due to
many successful operations is a rare achievement in pure random
testing.  For software where the state of the system is mostly a
matter of the contents of volatile memory, saving state is difficult
and many methods require access to source code \cite{ModelDriven}.  For ArcPy,
however, the most important system state is the changes made to
feature classes stored in non-volatile memory.

The ArcPy test harness already makes use of a set of ``seed'' files as
an initial state for testing.  There is no reason that all test
sequences must start from the same set of data, however.  The
flexibility that allows users to test ArcPy over their own data also
makes it possible to start testing from files resulting from previous
testing, without repeating the operations involved, trading storage
space (to hold the modified files) for testing runtime.  Simply
copying the workspace after each test completes and randomly selecting a
previously produced workspace to begin testing with is trivial to
implement.  However, this naive approach has two problems:

\begin{enumerate}
\item First, if a fault is detected using a complex starting state,
  the fault may be due to the actions that produced that state, not
  test actions that follow in the ``new'' test.  Providing only the
  seed files and the actions may make it very hard to understand and
  debug the fault.

\item Second, how do we determine which workspaces to store?  In many
  cases, due to the high probability of operation failure, the files
  involved will be unchanged, or only changed in small, uninteresting
  ways, after hundreds of actions.  Storing so many copies of the
  feature classes used in testing is highly inefficient.
\end{enumerate}

We propose combining our approach to producing regression tests with an
analysis of file contents in order to 1) filter out which workspace states to
store and 2) capture the action sequence (a test prefix) that produces a given workspace.
This not only enables debugging (including test case reduction,
normalization, and generalization) of test cases that start from a
saved \emph{deep state}, but allows a workspace to be ``zipped'' for download by only
providing the action sequence to produce it.

Our efforts along these lines are preliminary.  However, we believe
that a thorough investigation of the possibilities of saving system
state after long test sequences and starting testing from complex
states is a promising direction for future research.  Recent work has
shown that test startup costs are very high, relative to the actual
tested behavior \cite{Bell}.


\section{Faults Discovered}

\begin{figure}
{\scriptsize 
\begin{code}
shapefilelist0 = glob.glob("C:\\Arctmp\\*.shp")                             \textcolor{black!60}{\# STEP 0}
\textcolor{black!60}{\#[}
shapefile0 = shapefilelist0 [0]                                           \textcolor{black!60}{\# STEP 1}
newlayer0 = "l1"                                                          \textcolor{black!60}{\# STEP 2}
\textcolor{black!60}{\#  or newlayer0 = "l2" }
\textcolor{black!60}{\#  or newlayer0 = "l3" }
\textcolor{black!60}{\#  swaps with steps 3 4 5 6 7}
\textcolor{black!60}{\#] (steps in [] can be in any order)}
\textcolor{black!60}{\#[}
featureclass0 = shapefile0                                                \textcolor{black!60}{\# STEP 3}
\textcolor{black!60}{\#  swaps with step 2}
fieldname0 = "newf1"                                                      \textcolor{black!60}{\# STEP 4}
\textcolor{black!60}{\#  or fieldname0 = "newf2" }
\textcolor{black!60}{\#  or fieldname0 = "newf3" }
\textcolor{black!60}{\#  swaps with steps 2 8}
selectiontype0 = "SWITCH\_SELECTION"                                       \textcolor{black!60}{\# STEP 5}
\textcolor{black!60}{\#  or selectiontype0 = "NEW\_SELECTION" }
\textcolor{black!60}{\#  or selectiontype0 = "ADD\_TO\_SELECTION" }
\textcolor{black!60}{\#  or selectiontype0 = "REMOVE\_FROM\_SELECTION"}
\textcolor{black!60}{\#  or selectiontype0 = "SUBSET\_SELECTION"}
\textcolor{black!60}{\#  or selectiontype0 = "CLEAR\_SELECTION"   }
\textcolor{black!60}{\#  swaps with steps 2 8}
op0 = ">"                                                                 \textcolor{black!60}{\# STEP 6}
\textcolor{black!60}{\#  or op0 = "<" }
\textcolor{black!60}{\#  swaps with steps 2 8}
val0 = "100"                                                              \textcolor{black!60}{\# STEP 7}
\textcolor{black!60}{\#  or val0 = "1000" }
\textcolor{black!60}{\#  swaps with steps 2 8}
\textcolor{black!60}{\#] (steps in [] can be in any order)}
arcpy.MakeFeatureLayer\_management(featureclass0, newlayer0)               \textcolor{black!60}{\# STEP 8}
\textcolor{black!60}{\#  swaps with steps 4 5 6 7}
arcpy.SelectLayerByAttribute\_management(newlayer0,selectiontype0,
   ' "'+fieldname0+'" '+op0+val0)                                         \textcolor{black!60}{\# STEP 9}
arcpy.Delete\_management(featureclass0)                                    \textcolor{black!60}{\# STEP 10}
arcpy.SelectLayerByAttribute\_management(newlayer0,selectiontype0,
   ' "'+ fieldname0+'" '+op0+val0)                                        \textcolor{black!60}{\# STEP 11}
\end{code}
}
\caption{Deleting a feature class does not invalidate or delete layers that depend on it.}
\label{fault1}
\end{figure}

\begin{figure}
{\scriptsize 
\begin{code}
shapefilelist0 = sorted(glob.glob(arcpy.env.workspace + "\\*.shp"))            \textcolor{black!60}{\# STEP 0}
\textcolor{black!60}{\#[}
shapefile0 = shapefilelist0 [0]                                               \textcolor{black!60}{\# STEP 1}
newlayer0 = "l1"                                                              \textcolor{black!60}{\# STEP 2}
\textcolor{black!60}{\#  or newlayer0 = "l2" }
\textcolor{black!60}{\#  or newlayer0 = "l3" }
\textcolor{black!60}{\#  swaps with step 3}
\textcolor{black!60}{\#] (steps in [] can be in any order)}
\textcolor{black!60}{\#[}
featureclass0 = shapefile0                                                    \textcolor{black!60}{\# STEP 3}
\textcolor{black!60}{\#  swaps with step 2}
classorlayer0 = newlayer0                                                     \textcolor{black!60}{\# STEP 4}
\textcolor{black!60}{\#  swaps with steps 10 11 12}
fieldtype0 = "DATE"                                                           \textcolor{black!60}{\# STEP 5}
\textcolor{black!60}{\#  or fieldtype0 = "TEXT" }
\textcolor{black!60}{\#  or fieldtype0 = "FLOAT" }
\textcolor{black!60}{\#  or fieldtype0 = "DOUBLE" }
\textcolor{black!60}{\#  or fieldtype0 = "SHORT" }
\textcolor{black!60}{\#  or fieldtype0 = "LONG" }
\textcolor{black!60}{\#  swaps with steps 10 11 12}
fieldname0 = "newf1"                                                          \textcolor{black!60}{\# STEP 6}
\textcolor{black!60}{\#  swaps with steps 10 11 12}
op0 = ">"                                                                     \textcolor{black!60}{\# STEP 7}
\textcolor{black!60}{\#  or op0 = "<" }
\textcolor{black!60}{\#  or op0 = "<=" }
\textcolor{black!60}{\#  or op0 = ">=" }
\textcolor{black!60}{\#  or op0 = "=" }
\textcolor{black!60}{\#  swaps with steps 10 11 12 14}
val0 = "100"                                                                  \textcolor{black!60}{\# STEP 8}
\textcolor{black!60}{\#  or val0 = "1000" }
\textcolor{black!60}{\#  swaps with steps 10 11 12 14}
stattable0 = arcpy.env.workspace + "\\stats.dbf"                               \textcolor{black!60}{\# STEP 9}
\textcolor{black!60}{\#  swaps with steps 10 11 12 14}
\textcolor{black!60}{\#] (steps in [] can be in any order)}
\textcolor{black!60}{\#[}
fieldlist0 = arcpy.ListFields(featureclass0)                                  \textcolor{black!60}{\# STEP 10}
\textcolor{black!60}{\#  swaps with steps 4 5 6 7 8 9 14}
stattype0 = "FIRST"                                                           \textcolor{black!60}{\# STEP 11}
\textcolor{black!60}{\#  or stattype0 = "LAST" }
\textcolor{black!60}{\#  swaps with steps 4 5 6 7 8 9}
statfields0 = []                                                              \textcolor{black!60}{\# STEP 12}
\textcolor{black!60}{\#  swaps with steps 4 5 6 7 8 9}
\textcolor{black!60}{\#] (steps in [] can be in any order)}
\textcolor{black!60}{\#[}
statfields0.append([fieldname0,stattype0])                                    \textcolor{black!60}{\# STEP 13}
arcpy.AddField\_management(featureclass0,fieldname0,fieldtype0); report()      \textcolor{black!60}{\# STEP 14}
\textcolor{black!60}{\#  swaps with steps 7 8 9 10}
\textcolor{black!60}{\#] (steps in [] can be in any order)}
fieldname0 = fieldlist0 [0].name \textcolor{black!60}{\# STEP 15}
arcpy.MakeFeatureLayer\_management(featureclass0,newlayer0,
   where\_clause=' "' + fieldname0 + '" ' + op0 + val0); report()              \textcolor{black!60}{\# STEP 16}
fieldname0 = "newf1"                                                          \textcolor{black!60}{\# STEP 17}
arcpy.DeleteField\_management(featureclass0,fieldname0); report()              \textcolor{black!60}{\# STEP 18}
arcpy.Statistics\_analysis(classorlayer0,stattable0,statfields0); report()     \textcolor{black!60}{\# STEP 19}
\end{code}
}
\caption{Deleting a field then computing statistics on it causes a crash.}
\label{fault2}
\end{figure}

\begin{figure}
{\scriptsize 
\begin{code}
shapefilelist0 = sorted(glob.glob(arcpy.env.workspace + "\\*.shp"))         \textcolor{black!60}{\# STEP 0}
shapefile0 = shapefilelist0 [0]                                            \textcolor{black!60}{\# STEP 1}
featureclass0 = shapefile0                                                 \textcolor{black!60}{\# STEP 2}
\textcolor{black!60}{\#[}
classorlayer0 = featureclass0                                              \textcolor{black!60}{\# STEP 3}
fieldtype0 = "DOUBLE"                                                      \textcolor{black!60}{\# STEP 4}
\textcolor{black!60}{\#  or fieldtype0 = "TEXT"}
\textcolor{black!60}{\#  or fieldtype0 = "FLOAT"}
\textcolor{black!60}{\#  or fieldtype0 = "SHORT"}
\textcolor{black!60}{\#  or fieldtype0 = "LONG"}
\textcolor{black!60}{\#  or fieldtype0 = "DATE"}
\textcolor{black!60}{\#  swaps with step 6}
fieldname0 = "newf1"                                                       \textcolor{black!60}{\# STEP 5}
\textcolor{black!60}{\#  or fieldname0 = "newf3"}
\textcolor{black!60}{\#  swaps with steps 6 8}
\textcolor{black!60}{\#] (steps in [] can be in any order)}
\textcolor{black!60}{\#[}
insertcursor0 = arcpy.InsertCursor(classorlayer0)                          \textcolor{black!60}{\# STEP 6}
\textcolor{black!60}{\#  swaps with steps 4 5}
arcpy.AddField\_management(featureclass0,fieldname0,fieldtype0); report()   \textcolor{black!60}{\# STEP 7}
\textcolor{black!60}{\#] (steps in [] can be in any order)}
fieldname0 = "newf2"                                                       \textcolor{black!60}{\# STEP 8}
\textcolor{black!60}{\#  or fieldname0 = "newf3"}
\textcolor{black!60}{\#  swaps with step 5}
arcpy.AddField\_management(featureclass0,fieldname0,fieldtype0); report()   \textcolor{black!60}{\# STEP 9}
insertcursor0 = arcpy.InsertCursor(classorlayer0)                          \textcolor{black!60}{\# STEP 10}
\end{code}
}
\caption{Creating two insert cursors on a layer, after adding two fields to the feature class underlying it, causes a crash.}
\label{fault3}
\end{figure}

\begin{figure}
{\scriptsize
\begin{code}
shapefilelist0 = sorted(glob.glob(arcpy.env.workspace + "\\*.shp"))         \textcolor{black!60}{\# STEP 0}
\textcolor{black!60}{\#[}
shapefile0 = shapefilelist0 [0]                                            \textcolor{black!60}{\# STEP 1}
newlayer0 = "l1"                                                           \textcolor{black!60}{\# STEP 2}
\textcolor{black!60}{\#  or newlayer0 = "l2" }
\textcolor{black!60}{\#  swaps with step 3}
\textcolor{black!60}{\#] (steps in [] can be in any order)}
\textcolor{black!60}{\#[}
featureclass0 = shapefile0                                                 \textcolor{black!60}{\# STEP 3}
\textcolor{black!60}{\#  swaps with step 2}
classorlayer0 = newlayer0                                                  \textcolor{black!60}{\# STEP 4}
\textcolor{black!60}{\#  or classorlayer0 = featureclass0 }
\textcolor{black!60}{\#  or (}
\textcolor{black!60}{\#      newlayer0 = "l3"  ;}
\textcolor{black!60}{\#      classorlayer0 = newlayer0 }
\textcolor{black!60}{\#     )}
fieldtype0 = "FLOAT"                                                       \textcolor{black!60}{\# STEP 5}
\textcolor{black!60}{\#  or fieldtype0 = "TEXT" }
\textcolor{black!60}{\#  or fieldtype0 = "DOUBLE" }
\textcolor{black!60}{\#  or fieldtype0 = "SHORT" }
\textcolor{black!60}{\#  or fieldtype0 = "LONG" }
\textcolor{black!60}{\#  or fieldtype0 = "DATE" }
fieldname0 = "newf1"                                                       \textcolor{black!60}{\# STEP 6}
\textcolor{black!60}{\#  or fieldname0 = "newf3" }
\textcolor{black!60}{\#  swaps with step 11}
op0 = ">"                                                                  \textcolor{black!60}{\# STEP 7}
\textcolor{black!60}{\#  or op0 = "<" }
\textcolor{black!60}{\#  or op0 = "<=" }
\textcolor{black!60}{\#  or op0 = ">=" }
\textcolor{black!60}{\#  or op0 = "=" }
val0 = "10"                                                                \textcolor{black!60}{\# STEP 8}
\textcolor{black!60}{\#  or val0 = "20" }
\textcolor{black!60}{\#  or val0 = "30" }
\textcolor{black!60}{\#  or val0 = "100" }
\textcolor{black!60}{\#  or val0 = "1000" }
\textcolor{black!60}{\#] (steps in [] can be in any order)}
\textcolor{black!60}{\#[}
whereclause0 = '"' + fieldname0 + '" ' + op0 + str(val0)                   \textcolor{black!60}{\# STEP 9}
arcpy.AddField\_management(featureclass0,fieldname0,fieldtype0); report()   \textcolor{black!60}{\# STEP 10}
\textcolor{black!60}{\#] (steps in [] can be in any order)}
\textcolor{black!60}{\#[}
fieldname0 = "newf2"                                                       \textcolor{black!60}{\# STEP 11}
\textcolor{black!60}{\#  or fieldname0 = "newf3" }
\textcolor{black!60}{\#  swaps with step 6}
arcpy.MakeFeatureLayer\_management(featureclass0,newlayer0,where\_clause=whereclause0);
   report()                                                                \textcolor{black!60}{\# STEP 12}
\textcolor{black!60}{\#] (steps in [] can be in any order)}
searchcursor0 = arcpy.SearchCursor(classorlayer0,whereclause0)             \textcolor{black!60}{\# STEP 13}
\textcolor{black!60}{\#  or searchcursor0 = arcpy.SearchCursor(classorlayer0) }
arcpy.AddField\_management(featureclass0,fieldname0,fieldtype0); report()   \textcolor{black!60}{\# STEP 14}
srow0 = searchcursor0.next()                                               \textcolor{black!60}{\# STEP 15}
\textcolor{black!60}{\#  or srow1 = searchcursor0.next() }
\textcolor{black!60}{\#  or srow2 = searchcursor0.next()}
\end{code}
}
\caption{Advancing a search cursor on a layer, after adding a field to
  the underlying feature class twice, causes a crash.}
\label{fault4}
\end{figure}

Thus far, our fault-finding process has focused on crashes.  Because
test cases that cause crashes stop the testing process, it is critical
to identify and avoid behavior that causes crashes before proceeding
to add further correctness properties.  Additionally, other than
hard-to-detect data corruption, crashes may be the most frustrating
faults for a developer to fix.  When ArcGIS or a standalone ArcPy
script causes a system crash, there is no readable error message, or
symptom in incorrect data to use in debugging the program.
Understanding system core dumps or analyzing the XML logs produced by
the ArcGIS engine is difficult for end users.  It would be ideal to
fix all ArcPy crashes by (at least) changing the library behavior to
issue an error message on invalid calls, but in the absence of bug
fixes, it is helpful to identify the root causes of various crashes
for users.  These test cases all involve calls that, as far as we
can tell, are not forbidden by ArcPy documentation.

Figures \ref{fault1}-\ref{fault4} show four test cases that result in
an ArcPy crash (the Python interpreter stops functioning, terminating
the test prematurely).  If executed in ArcGIS, these fragments will
crash ArcGIS.  Because these test cases are 1-minimal \cite{DD},
normalized, and generalized \cite{ICSTnorm}, we are able to describe in some detail
the general sequence of actions that produces each crash.  We have
discovered a few other crash faults; these are either not yet
well understood, or (we believe) equivalent in underlying cause
to these crashes.  The crashes discovered also serve as a basic proof
of concept that ArcPy test generation works and discovers
unanticipated interactions of API calls in the system.

\subsection{First Crash Fault}

ArcPy crashes when the feature class from which a layer is produced is
deleted, and the layer is used in a {\tt SelectLayer} call (this
version shows an attribute-based selection, but location selection
will cause the same problem): (Figure \ref{fault1}).  The underlying issue seems to be that
while operations on a deleted feature class properly notify a user the
feature class does not exist, ArcPy or ArcGIS does not track that
layers depending on that feature should also be deleted/invalidated
when the feature class is deleted.  Layers are not copies
of a feature class, but essentially new views of a feature class.
This means that when the underlying feature class is modified or
deleted, the view needs to be updated to reflect that change, and this
is not always correctly implemented.


\subsection{Second Crash Fault}

ArcPy crashes when asked to compute statistics (wth the {\tt FIRST} or
{\tt LAST} statistics types) over a field of a layer, when that field
has been deleted from the underlying feature class:  Figure
\ref{fault2}.  This is possibly related to the first crash fault:
ArcGIS again does not seem to properly propagate changes to an underlying
feature class to layers (which seem to be views) created on that feature class using a {\tt
  MakeFeatureLayer} call.



\subsection{Third and Fourth Crash Faults}

Creating an insert cursor on a feature class, before or after adding
one field to the feature class, then adding a second new field, then
creating a second insert cursor, causes ArcPy to crash:  Figure
\ref{fault3}.  A similar, but not obviously equivalent problem is
shown in Figure \ref{fault4}, where the combination and cursor type
are different, but the same issue of cursors interacting with feature
class or layer changes appears.  It seems likely that ArcPy should
simply add a requirement that feature classes or layers with active
cursors should not be modified at all, except by the active cursor.


%\subsection{Deep State Startup Testing}

One of the most challenging problems in testing ArcPy is that
individual test operations may take much more time than in traditional
API-based testing.  In traditional API testing, it is assumed that
each step of execution takes at most a few seconds.  Performing a
complex GIS analysis such as a Buffer or Intersect, however, can require
many seconds, or even many minutes (the largest time for a single
operation during a test we have seen is almost 30 minutes).  If some faults depend
on the interactions of multiple complex analyses, detecting them will
be extremely expensive.  Simply reaching system states that are due to
many successful operations is a rare achievement in pure random
testing.  For software where the state of the system is mostly a
matter of the contents of volatile memory, saving state is difficult
and many methods require access to source code \cite{ModelDriven}.  For ArcPy,
however, the most important system state is the changes made to
feature classes stored in non-volatile memory.

The ArcPy test harness already makes use of a set of ``seed'' files as
an initial state for testing.  There is no reason that all test
sequences must start from the same set of data, however.  The
flexibility that allows users to test ArcPy over their own data also
makes it possible to start testing from files resulting from previous
testing, without repeating the operations involved, trading storage
space (to hold the modified files) for testing runtime.  Simply
copying the workspace after each test completes and randomly selecting a
previously produced workspace to begin testing with is trivial to
implement.  However, this naive approach has two problems:

\begin{enumerate}
\item First, if a fault is detected using a complex starting state,
  the fault may be due to the actions that produced that state, not
  test actions that follow in the ``new'' test.  Providing only the
  seed files and the actions may make it very hard to understand and
  debug the fault.

\item Second, how do we determine which workspaces to store?  In many
  cases, due to the high probability of operation failure, the files
  involved will be unchanged, or only changed in small, uninteresting
  ways, after hundreds of actions.  Storing so many copies of the
  feature classes used in testing is highly inefficient.
\end{enumerate}

We propose combining our approach to producing regression tests with an
analysis of file contents in order to 1) filter out which workspace states to
store and 2) capture the action sequence (a test prefix) that produces a given workspace.
This not only enables debugging (including test case reduction,
normalization, and generalization) of test cases that start from a
saved \emph{deep state}, but allows a workspace to be ``zipped'' for download by only
providing the action sequence to produce it.

Our efforts along these lines are preliminary.  However, we believe
that a thorough investigation of the possibilities of saving system
state after long test sequences and starting testing from complex
states is a promising direction for future research.  Recent work has
shown that test startup costs are very high, relative to the actual
tested behavior \cite{Bell}.
\section{Work In-Progress}
\label{future}

While not all of the language and tool changes discussed above have
been introduced in the release version of TSTL, all of these features are at
minimum undergoing testing prior to release.  The features discussed
in this section are in the early, exploratory, design phase.

\subsection{More Complete Correctness Checks}
\label{futureoracle}

One key omission in the ArcPy test harness is the lack of correctness
properties.  This version only fails tests when 1) the system crashes
or 2) an unexpected exception is raised.  Determining the proper
results for complex operation sequences on arbitrary data is
difficult.  To support conditional metamorphic properties \cite{MetaTest}, we plan to
add a TSTL feature to check a property conditional on the set of
actions called.  For example, if fields are never added or deleted, the number of
fields for each feature class should remain constant.  Unfortunately,
the set of properties that are independent of the underlying data used
may be relatively small, or the conditions under which a property
should hold so limited it may be hard to produce highly effective
testing based just on conventional correctness properties or limited
metamorphic testing.

\subsection{Differential Testing within One Version}
\label{sec:reftest}

One way to work around the difficulties of specifying properties of
complex systems is to use differential testing \cite{Differential,ICSEDiff}:  if two systems
implement the same functionality, and are given the same inputs, they
should produce the same output.  This approach can also be applied
within a single version of ArcPy:  because ArcGIS supports more than
one storage format (shapefiles, file geodatabases, and personal
geodatabases), the same data can be stored in multiple formats.  Using
TSTL's support for reference pools, the same operations can be
performed on the same data stored in different formats, and the
results can be checked, including success or failure of ArcPy calls
and spatial properties and field values of the manipulated data.

\subsection{Testing Multiprocessing}

ArcPy scripts can use Python's multiprocessing features to exploit
multicore machines and gain efficiency.  However, this can expose
ArcGIS or ArcPy faults, especially problems with the (not well
documented) locking system.  TSTL can use multiprocessing in actions
(since actions can call arbitrary Python code), but does not include
support for running multiple threads of execution at the TSTL level.
One problem with testing multiprocessing use of ArcPy is that
concurrency is much more likely to produce nondeterministic test
behavior than our existing testing.

%\subsection{Optimizations for Testing}

\subsection{Parallel Testing}

While multiprocessing on a single machine can introduce additional
problems with test determinism and locking interactions, running tests
in parallel using either cloud computing or virtualization on a
single multicore machine could improve test throughput.   This is
possibly more important for ArcPy than in many other testing
scenarios, given the unusually high cost of even single API calls.

\subsection{Alternative Test Generation Techniques}

TSTL is meant to provide a platform for experimenting with various
methods for generating tests \cite{NFM15}.  To date, we have only used
pure random testing, since for our purposes simply generating tests as
quickly and simply as possible has been highly effective.  However, in
the future we plan to explore alternative generation methods.  In some
cases, this should be simple.  For example, swarm testing
\cite{ISSTA12} only requires identifying features \cite{groce2013help}
(which for ArcPy will simply be different API functions) that can be
omitted from tests, to concentrate testing on a particular aspect of
the API.  Various search-based \cite{FA11} and machine-learning-based
\cite{ISSRE} approaches to test case generation can also be implemented
  using TSTL.  The inabilty to save ArcPy memory state and the high
  cost of backtracking via replay makes methods such as BFS and DFS
  exploration \cite{NFM15} that resemble model-checking
  \cite{ModelChecking,ModelDriven} unlikely to pay off, however.

\section{The ArcPy TSTL Test Harness}
\label{harness}

\begin{figure}
{\scriptsize
\begin{code}
@import shutil, os, glob, arcpy, exceptions, gc
@from arcpy import ExecuteError
<@
def cleanupFiles():
    gc.collect() \# Get rid of cursors
    for l in ["l1","l2","l3"]:
    	arcpy.Delete\_management(l)
    
    for f in glob.glob("C:\\Arctmp\\*"):
        try:
            shutil.rmtree(f)
        except:
            print "UNABLE TO REMOVE:",f
    for i in xrange(0,1000000): \# Find a workspace without a lock
        new\_workspace = "C:\\Arctmp\\workspace." + str(i)
        if not os.path.exists(new\_workspace):
            break             
    shutil.copytree("C:\\Arcbase",new\_workspace)
    arcpy.env.workspace = new\_workspace
    print sorted(glob.glob(arcpy.env.workspace + "\\*.shp")),
    print sorted(glob.glob(arcpy.env.workspace + "\\*.lyr")),
    print sorted(glob.glob(arcpy.env.workspace + "\\*.gdb"))


def fcsInGdb(gdb):
    old\_workspace = arcpy.env.workspace
    arcpy.env.workspace = gdb
    fcs = []
    for fds in arcpy.ListDatasets('','feature') + ['']:
        for fc in arcpy.ListFeatureClasses('','',fds):
            fcs.append(os.path.join(gdb,fds,fc))
    arcpy.env.workspace = old\_workspace
    return fcs

def report():
    print arcpy.GetMessages()
@>

pool: <shapefilelist> 2
pool: <shapefile> 2 CONST

pool: <gdbfilelist> 2
pool: <gdbfile> 2 CONST

pool: <gdbfeatureclasslist> 2

pool: <featureclass> 4 CONST
pool: <classorlayer> 4 CONST

pool: <classorlayerlist> 2

pool: <spatialref> 2

pool: <prjfilelist> 2
pool: <prjfile> 2 CONST

pool: <transformlist> 2
pool: <transform> 2 CONST

pool: <newlayer> 2 CONST

pool: <layerlist> 2
pool: <layer> 2

pool: <fieldname> 2 CONST
pool: <fieldtype> 2 CONST
pool: <fieldlist> 2
pool: <fieldnamelist> 2

pool: <stattype> 2 CONST
pool: <statfields> 2

pool: <dist> 2 CONST

pool: <sorttype> 2 CONST
pool: <spatialsort> 2 CONST

pool: <sort> 1
pool: <sortlist> 2

pool: <joinattributes> 2

pool: <overlaptype> 2 CONST
pool: <selectiontype> 2 CONST
pool: <op> 2 CONST
pool: <val> 2 CONST
pool: <whereclause> 2 CONST

pool: <errtable> 1 CONST
pool: <polytable> 1 CONST
pool: <stattable> 1 CONST

pool: <insertcursor> 3
pool: <searchcursor> 3
pool: <updatecursor> 3

pool: <irow> 3
pool: <srow> 3
pool: <urow> 3

init: cleanupFiles()

log: 1 arcpy.GetMessages()
\end{code}
}
\caption{ArcPy TSTL test harness definition preamble (pools, functions, logging).}
\label{preamble}
\end{figure}

\begin{figure}
{\scriptsize
\begin{code}
<gdbfilelist> := sorted(glob.glob(arcpy.env.workspace + "\\*.gdb"))
len(<gdbfilelist,1>) >= 1 -> <gdbfile> := <gdbfilelist> [0]
<gdbfilelist> = <gdbfilelist> [1:]

<shapefilelist> := sorted(glob.glob(arcpy.env.workspace + "\\*.shp"))
len(<shapefilelist,1>) >= 1 -> <shapefile> := <shapefilelist> [0]
<shapefilelist> = <shapefilelist> [1:]

<layerfile> := arcpy.env.workspace + <["\\new1.lyr", "\\new2.lyr", "\\new3.lyr"]>

<shapefile> := arcpy.env.workspace + <["\\new1.shp", "\\new2.shp", "\\new3.shp"]>

<prjfilelist> := sorted(glob.glob
   ("C:\\Program Files (x86)\\ArcGIS\\Desktop10.3\\Reference Systems\\*.prj"))
len(<prjfilelist,1>) >= 1 -> <prjfile> := <prjfilelist> [0]
<prjfilelist> = <prjfilelist> [1:]

<transformlist> := arcpy.ListTransformations(<spatialref>,<spatialref>)
<transformlist> = <transformlist> [1:]
len(<transformlist,1>) >= 1 -> <transform> := <transformlist> [0]

<newlayer> := <["l1", "l2", "l3"]>

<spatialref> := arcpy.SpatialReference(<prjfile>)

<gdbfeatureclasslist> := fcsInGdb(<gdbfile>)
<gdbfeatureclasslist> = <gdbfeatureclasslist>[1:]

<featureclass> := <shapefile>
len(<gdbfeatureclasslist,1>) >= 1 -> <featureclass> := <gdbfeatureclasslist>[0]

<classorlayer> := <featureclass>
<classorlayer> := <newlayer>

<classorlayerlist> := []
<classorlayerlist>.append(<classorlayer>)

<fieldtype> := <["TEXT", "FLOAT", "DOUBLE", "SHORT", "LONG", "DATE"]>

<fieldname> := <["newf1", "newf2", "newf3"]>

<dist> := <["100 Feet", "500 Feet", "1000 Feet", "1 Mile", "2 Miles">]

<joinattributes> := <["ALL", "NO\_FID", "ONLY\_FID"]>

<overlaptype> := <["INTERSECT", "CONTAINS", "COMPLETELY\_CONTAINS", "WITHIN",
   "SHARE\_A\_LINE\_SEGMENT\_WITH", "CROSSED\_BY\_THE\_OUTLINE\_OF"]>

<selectiontype> := <["NEW\_SELECTION", "ADD\_TO\_SELECTION", "REMOVE\_FROM\_SELECTION",
   "SUBSET\_SELECTION", "SWITCH\_SELECTION", "CLEAR\_SELECTION"]>

<sorttype> := <["ASCENDING", "DESCENDING">]

<spatialsort> := <["UR", "UL", "LR", "LL", "PEANO"]>

<sort> := [<fieldname>,<sorttype>]
<sortlist> := []
<sortlist>.append(<sort>)

\{IOError\} <insertcursor> := arcpy.InsertCursor(<classorlayer>)
\{IOError\} <insertcursor> := arcpy.InsertCursor(<classorlayer>,<spatialref>)
\{IOError\} <searchcursor> := arcpy.SearchCursor(<classorlayer>)
\{IOError,exceptions.RuntimeError\} <searchcursor> := arcpy.SearchCursor
   (<classorlayer>,<whereclause>)
\{IOError,exceptions.RuntimeError\} <searchcursor> := arcpy.SearchCursor
   (<classorlayer>,<whereclause>,<spatialref>)
\{IOError\} <updatecursor> := arcpy.UpdateCursor(<classorlayer>)
\{IOError\} <updatecursor> := arcpy.UpdateCursor(<classorlayer>,<spatialref>)

<irow> := <insertcursor>.newRow()
\{exceptions.RuntimeError\} <insertcursor>.insertRow(<irow>)

<irow> := <insertcursor>.next()
<urow> := <updatecursor>.next()
<srow> := <searchcursor>.next()

\{exceptions.RuntimeError\} <val> := <irow>.getValue(<fieldname>)
\{exceptions.RuntimeError\} <val> := <srow>.getValue(<fieldname>)
\{exceptions.RuntimeError\} <val> := <urow>.getValue(<fieldname>)

\{exceptions.RuntimeError\} <irow>.setValue(<fieldname>,<val>)

\{exceptions.RuntimeError\} <urow>.setNull(<fieldname>)
\{exceptions.RuntimeError\} <urow>.setValue(<fieldname>,<val>)
\{exceptions.RuntimeError\} <updatecursor>.deleteRow(<urow>)
\{exceptions.RuntimeError\} <updatecursor>.updateRow(<urow>)

<op> := <[">", "<", "<=", ">=", "=", "!=">]  

<val> := <["10", "20", "30", "100", "1000">]

<whereclause> := '"' + <fieldname> + '" ' + <op> + str(<val>)

<whereclause> := <whereclause> + ' AND ' + <whereclause>

<whereclause> := <whereclause> + ' OR ' +  <whereclause>

<whereclause> := 'NOT' + <whereclause>

<errtable> := arcpy.env.workspace + "\\geomerr.dbf"

<polytable> := arcpy.env.workspace + "\\polyneig.dbf"

<stattable> := arcpy.env.workspace + "\\stats.dbf"

\{IOError\} <fieldlist> := arcpy.ListFields(<classorlayer>)
len(<fieldlist,1>) >= 1 -> <fieldname> := <fieldlist> [0].name
<fieldlist> = <fieldlist> [1:]

<fieldnamelist> := []
<fieldnamelist>.append(<fieldname>)

<stattype> := <["SUM", "MEAN", "MIN", "MAX", "RANGE", "STD", "COUNT", "FIRST", "LAST"]>

<statfields> := []
<statfields>.append([<fieldname>,<stattype>])
\end{code}
}
\caption{ArcPy TSTL test harness actions, part 1.}
\label{actions1}
\end{figure}

\begin{figure}
{\scriptsize
\begin{code}
\{ExecuteError\} arcpy.MakeFeatureLayer\_management(<featureclass>,<newlayer>); report()

\{ExecuteError\} arcpy.MakeFeatureLayer\_management(<featureclass>,<newlayer>,
   where\_clause=<whereclause>); report()

\{ExecuteError\} arcpy.Project\_management(<featureclass>,<featureclass>,<spatialref>,
   <transform>); report()

\{ExecuteError\} arcpy.AddField\_management(<featureclass>,<fieldname>,<fieldtype>);
   report()

\{ExecuteError\} arcpy.DeleteField\_management(<featureclass>,<fieldname>); report()

\{ExecuteError\} arcpy.Buffer\_analysis(<classorlayer>,<featureclass>,<dist>); report()

\{ExecuteError\} arcpy.Buffer\_analysis(<classorlayer>,<featureclass>,<dist>,
   dissolve\_option="ALL"); report()

\{ExecuteError\} arcpy.Buffer\_analysis(<classorlayer>,<featureclass>,<dist>,
   dissolve\_option="LIST",dissolve\_field=<fieldnamelist>); report()

\{ExecuteError\} arcpy.Erase\_analysis(<classorlayer>,<classorlayer>,<featureclass>);
   report()

\{ExecuteError\} arcpy.Erase\_analysis(<classorlayer>,<classorlayer>,<featureclass>,
   cluster\_tolerance=<dist>); report()

\{ExecuteError\} arcpy.Intersect\_analysis(<classorlayerlist>,<featureclass>); report()

\{ExecuteError\} arcpy.Intersect\_analysis(<classorlayerlist>,<featureclass>,
   join\_attributes=<joinattributes>); report()

\{ExecuteError\} arcpy.Intersect\_analysis(<classorlayerlist>,<featureclass>,
   cluster\_tolerance=<dist>); report()

\{ExecuteError\} arcpy.Intersect\_analysis(<classorlayerlist>,<featureclass>,
   join\_attributes=<joinattributes>,cluster\_tolerance=<dist>); report()

\{ExecuteError\} arcpy.Union\_analysis(<classorlayerlist>,<featureclass>); report()

\{ExecuteError\} arcpy.Union\_analysis(<classorlayerlist>,<featureclass>,
   join\_attributes=<joinattributes>); report()

\{ExecuteError\} arcpy.Union\_analysis(<classorlayerlist>,<featureclass>,
   cluster\_tolerance=<dist>); report()

\{ExecuteError\} arcpy.Union\_analysis(<classorlayerlist>,<featureclass>,
   join\_attributes=<joinattributes>,cluster\_tolerance=<dist>); report()

\{ExecuteError\} arcpy.SpatialJoin\_analysis(<classorlayer>,<classorlayer>,
   <featureclass>); report()

\{ExecuteError\} arcpy.SymDiff\_analysis(<classorlayer>,<classorlayer>,<featureclass>);
   report()

\{ExecuteError\} arcpy.SymDiff\_analysis(<classorlayer>,<classorlayer>,<featureclass>,
   join\_attributes=<joinattributes>); report()

\{ExecuteError\} arcpy.SymDiff\_analysis(<classorlayer>,<classorlayer>,<featureclass>,
   join\_attributes=<joinattributes>,cluster\_tolerance=<dist>); report()

\{ExecuteError\} arcpy.PolygonNeighbors\_analysis(<classorlayer>,~<polytable>); report()

\{ExecuteError\} arcpy.Statistics\_analysis(<classorlayer>,~<stattable>,<statfields>);
   report()

\{ExecuteError\} arcpy.SelectLayerByLocation\_management(<newlayer>,
   select\_features=<newlayer>,overlap\_type=<overlaptype>)

\{ExecuteError\} arcpy.SelectLayerByLocation\_management(<newlayer>,
   select\_features=<newlayer>,overlap\_type=<overlaptype>,search\_distance=<dist>)

\{ExecuteError\} arcpy.SelectLayerByLocation\_management(<newlayer>,
   select\_features=<newlayer>,overlap\_type=<overlaptype>,search\_distance=<dist>,
   selection\_type=<selectiontype>)

\{ExecuteError\} arcpy.SelectLayerByAttribute\_management(<newlayer>,
   selection\_type=<selectiontype>,where\_clause=<whereclause>)

\{ExecuteError\} arcpy.Select\_analysis(<classorlayer>,<featureclass>,
   where\_clause=<whereclause>)

\{ExecuteError\} arcpy.CopyFeatures\_management(<featureclass>,<featureclass>); report()

\{ExecuteError\} arcpy.Sort\_management(<featureclass>,<featureclass>,<sortlist>);
   report()

\{ExecuteError\} arcpy.Sort\_management(<featureclass>,<featureclass>,<sortlist>,
   <spatialsort>); report()

\{ExecuteError\} arcpy.Sort\_management(<featureclass>,<featureclass>,
   [["Shape",<sorttype>]],<spatialsort>); report()

\{ExecuteError\} arcpy.CheckGeometry\_management(<classorlayer>,~<errtable>); report()

\{ExecuteError\} arcpy.CheckGeometry\_management(<classorlayerlist>,~<errtable>); report()

\{ExecuteError\} arcpy.Delete\_management(<featureclass>); report()
\end{code}
}
\caption{ArcPy TSTL test harness definition actions, part 2.}
\label{actions2}
\end{figure}



Figures \ref{preamble}-\ref{actions2} show a
version of the actual ArcPy test harness.  This version can reproduce the faults
described in this paper, though in practice some faults are easier to
detect than others, and for practical testing it is best to disable,
e.g., the various {\tt Delete} calls and to use guards to prevent all modifications of
layers or classes on which cursors are active.  Compiled to Python,
this harness defines nearly 2,000 actions, and the standalone
interface is nearly 60KLOC.  Using the harness involves first
compiling it to a Python class, then loading a test case generator
such as the random tester provided with TSTL into a Python environment
that has access to ArcPy.  For our experiments, we compiled the TSTL
using the Cygwin Python installation (for easy command-line access) but ran
tests in the IDLE environment installed with ArcGIS.
Running tests involves no complications beyond modifying parameters of
the random tester, if desired (setting a time limit for testing, the
length of test cases \cite{ASE08}, and whether to search for failures or produce
coverage regression tests, for example).  In most cases, the random tester
eventually terminates abnormally, and the test case causing the crash
is stored in a file in the ArcPy harness directory.  For regression
generation, the tool produces files of the same format to obtain
coverage of ArcPy code.

Figure \ref{preamble} contains the small amount of code needed to
prepare a temporary workspace for each test sequence.  This code
handles deleting any live cursors (the pool variables are guaranteed
to be deleted before each test, and garbage collection ensures the
cursors are deactivated), removing any layers created on feature
classes, and then setting the environment.  The system will scan for a
``free'' environment location, to handle cases where locks prevent
re-using an old directory\footnote{The problem of locked directories
  seldom surfaces, now that layers are deleted before each test, but
  may be needed when saving states.}.  The other utility functions
allow discovering the feature classes in a geodatabase and produce a
report on screen when a complex ArcGIS operation completes
successfully.

Figure \ref{actions1} shows the actions that create
input parameters for ArcGIS engine calls, primarily.  Many of these
simply pick some numeric or string constant (and the sets of constants
could be expanded, at the cost of more expensive normalization and
generalization).  Others, such as SQL query generation, are more
complex, with recursive expansion to theoretically unlimited query
length.  

Finally, Figure \ref{actions2} shows the TSTL for the actual
ArcGIS toolbox calls.  So far, this includes only a small subset of
the functions defined in the Management and Analysis toolboxes.  Note
that after each call there is a call to {\tt report}.  In pure random
testing, successful calls are infrequent enough that it is useful
for the user to see the ArcGIS messages produced by successful calls.
By turning on logging, the user can also see unsuccessful call
messages, but this tends to produce an overwhelming amount of output.

One critical design decision taken early in the development of this
harness was that there is no explicit definition of the starting data
used for testing.  Any data stored in the {\tt Arcbase} directory can
be used, and the harness supports both shapefiles and file geodatabases.
This has two purposes:  first, testing can be customized to use any
starting data, including a user's own shapefiles.  Second, this makes
it easy to implement deep state testing, since no assumptions are made
about the structure or number of data files used.  The lack of
dependence on specific files is implemented by having the test harness
use Python's {\tt glob} function to collect all files of a given
extension in a directory in a list, then chose an item from the list.
In order to make sure that behavior is deterministic (up to the choice
of actual data files), the glob results are sorted.  This approach
does bias random testing to using files earlier in alphabetical order,
but we do not expect base testing data to include a large number of
files.

The key driving requirements for this harness design are given in the
title of this paper:  the test harness must be \emph{extensible} and
it must be \emph{usable}.

Because ArcPy is large and complex, the effort to produce a
complete test harness and more effective specification of correctness
is a long-term effort, and may be carried out by other users:  the
harness therefore must be \emph{extensible}.  The
harness should also be suitable for use by ArcPy developers testing
how their own extensions to ArcPy interact with the base ArcPy API.
Adding new API calls to the harness should be easy:  we have defined
the pools for basic ArcPy object types and provided tools that should
handle much more complex test cases.  The decision discussed above to
make the test harness independent of specific data files is a good
example of our emphasis on extensibility.

The idea that other ArcPy developers, testing researchers, and perhaps
(end-user) software developers in other fields looking for a large
scale example of how to test systems with TSTL, will need to read and
modify the code also drives our emphasis on a truly usable system.
Usability has a more direct impact on the modifications to the TSTL
language and tools than on the test harness itself, but the test
harness is developed in the context of the improvements to the TSTL
ecosystem, such as standalone tests and normalization and
generalization.  However, some usability choices are decisions about
how to write the harness.  Consider the calls to {\tt
  SelectLayerByLocation\_management} shown in Figure \ref{actions2}.
The three lines of code could be more concisely expressed in one line using the {\tt <,[} construct:

\begin{code}
\{ExecuteError\} arcpy.SelectLayerByLocation\_management(
   <newlayer>,select\_features=<newlayer>,overlap\_type=
   <overlaptype><,[,search\_distance=<dist>,,],>
   <,[,selection\_type=<selectiontype>,,],>)
\end{code}

However, this is difficult to read, even for the TSTL developer, and
seems likely to discourage GIS developers trying to understand and
extend the harness.  There is a tension between, on the one hand, concise, abstracted
expression of all possible test actions and, on the other hand,
concrete readable connection between test actions and the lines of code that
appear in real ArcPy scripts.  We aim to stay closer to concrete
forms, even at some cost in increased length for the harness.  An
interesting observation is that this preference on the one hand makes
changing the harness more difficult --- to change the name of an API
call requires multiple edits if multiple parameter arities are used;
on the other hand, changing the code using {\tt <,[} requires
understanding the meaning of the code, each time, which resembles the
difficulty of altering a complex regular expression \cite{RegExp}.  In
this case, the best solution might be to add a specialized construct
to TSTL to handle optional elements of an action --- despite the fact
that the {\tt <,[} construct can express optional elements easily.
Even with such an option, it may be easier for developers to read and
understand the implications of individual calls, however, if they are
written out in the harness, rather than combinatorally generated by
the TSTL compiler.

In the long run, these issues are as complex as the questions of
abstraction vs. ease-of-understanding that have concerned designers
and users of programming languages since early in the history of
computer science.  As test definition comes closer to a specialized
kind of declarative programming, our understanding of the tradeoffs
will likely improve.
\section{Conclusions}
\label{conclusion}

This paper reports on an end-user-driven automated testing effort for a Python
library used to automate Geographic Information System analysis and
data management.  The test system is based on the TSTL
\cite{NFM15,ISSTA15,tstl} domain-specific language for testing, which
enables a declarative style of test harness development, where the
focus is on defining the actions in valid tests, not determining
exactly how tests are generated.

The complexity, size, and some unusual features of
the SUT have driven some engineering decisions, enhancements to the
TSTL language, and new testing utilities.  This paper focuses on
presenting the experience of testing a large complex system, and (we
hope) demonstrates that a domain expert whose programming experience
consists almost entirely of using the Software Under Test can make use
of modern automated test generation methods to find faults in complex software systems.


% BibTeX users please use one of
%\bibliographystyle{spbasic}      % basic style, author-year citations
\bibliographystyle{spmpsci}      % mathematics and physical sciences
%\bibliographystyle{spphys}       % APS-like style for physics
\bibliography{bibliography}   % name your BibTeX data base

% Non-BibTeX users please use
%\begin{thebibliography}{}
%
% and use \bibitem to create references. Consult the Instructions
% for authors for reference list style.
%
%\bibitem{RefJ}
% Format for Journal Reference
%Author, Article title, Journal, Volume, page numbers (year)
% Format for books
%\bibitem{RefB}
%Author, Book title, page numbers. Publisher, place (year)
% etc
%\end{thebibliography}

\end{document}
% end of file template.tex

